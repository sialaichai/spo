\documentclass[a4paper,12pt]{article}
\usepackage[utf8]{inputenc}
\usepackage[T1]{fontenc}
\usepackage{amsmath}
\usepackage{amssymb}
\usepackage{graphicx}
\usepackage{geometry}
\usepackage{siunitx}
\usepackage{fancyhdr}
\usepackage{tikz}

% Page Setup
\geometry{margin=1in}
\pagestyle{fancy}
\fancyhf{}
\lhead{SPhO 2018 Theory Paper 2 Solutions}
\rhead{\thepage}

% Spacing improvements for readability
\setlength{\parskip}{1em}      % Space between paragraphs
\setlength{\parindent}{0pt}    % No indentation for new paragraphs
\linespread{1.15}              % Slightly increased line spacing

\title{\textbf{31\textsuperscript{st} Singapore Physics Olympiad (SPhO) 2018} \\ \Large Theory Paper 2 Solutions}
\author{}
\date{}

\begin{document}

\maketitle
%\tableofcontents
%\newpage

% ==========================================
% QUESTION 1
% ==========================================
\section*{Question 1: Atomic Collisions}
\hrule
\vspace{0.5cm}

\subsection*{(a) Allowed Kinetic Energies}

\textbf{Problem Statement:}
A neutron ($m_n = m$, $K_n = \SI{68.2}{eV}$) collides with a stationary singly ionized Helium atom ($He^+$, mass $M = 4m$, ground state $n=1$).
After the collision:
\begin{itemize}
    \item The neutron scatters at $90^\circ$ (towards $-y$) with kinetic energy $K'_n$.
    \item The $He^+$ ion moves off at an angle and is excited to a higher quantum state $n$.
    \item $He^+$ energy levels: $E_n = -\frac{54.4}{n^2}$ eV.
\end{itemize}

\textbf{Solution:}

Let the initial direction of the neutron be the $+x$ direction.

\textbf{Initial state:}
\begin{itemize}
    \item Neutron: Momentum $\vec{p}_n = p \hat{i}$, Energy $K_n = \frac{p^2}{2m} = 68.2$ eV.
    \item Helium: Momentum $\vec{P}_{He} = 0$, Internal Energy $E_1 = -54.4$ eV.
\end{itemize}

\textbf{Final state:}
\begin{itemize}
    \item Neutron: Momentum $\vec{p}'_n = -p' \hat{j}$, Energy $K'_n = \frac{p'^2}{2m}$.
    \item Helium: Momentum $\vec{P}'_{He}$, Energy $K_{He} = \frac{P'^2_{He}}{2(4m)}$, Internal Energy $E_n$.
\end{itemize}

\vspace{0.5cm}
\textbf{1. Conservation of Momentum:}

\begin{align*}
    \vec{p}_n &= \vec{p}'_n + \vec{P}'_{He} \implies \vec{P}'_{He} = \vec{p}_n - \vec{p}'_n \\
    \vec{P}'_{He} &= p \hat{i} - (-p' \hat{j}) = p \hat{i} + p' \hat{j}
\end{align*}

The square of the magnitude of the Helium momentum is:
\[
P'^2_{He} = p^2 + p'^2
\]

\vspace{0.5cm}
\textbf{2. Kinetic Energy Relation:}

The kinetic energy of the recoiling Helium is:
\begin{align*}
    K_{He} &= \frac{P'^2_{He}}{2(4m)} = \frac{p^2 + p'^2}{8m} \\
    &= \frac{1}{4} \left( \frac{p^2}{2m} + \frac{p'^2}{2m} \right) \\
    &= \frac{1}{4} (K_n + K'_n)
\end{align*}

\vspace{0.5cm}
\textbf{3. Conservation of Energy:}

\begin{align*}
    K_{initial} + E_{internal, i} &= K_{final} + E_{internal, f} \\
    K_n + E_1 &= K'_n + K_{He} + E_n
\end{align*}

Let the excitation energy be $\Delta E = E_n - E_1$.
\begin{align*}
    K_n &= K'_n + \frac{1}{4}(K_n + K'_n) + \Delta E \\
    K_n - \frac{1}{4}K_n &= \frac{5}{4}K'_n + \Delta E \\
    \frac{3}{4}K_n &= \frac{5}{4}K'_n + \Delta E
\end{align*}

Multiplying by 4:
\begin{align*}
    3K_n &= 5K'_n + 4\Delta E \\
    5K'_n &= 3K_n - 4\Delta E \\
    K'_n &= \frac{3(68.2) - 4\Delta E}{5} = \frac{204.6 - 4\Delta E}{5}
\end{align*}

\vspace{0.5cm}
\textbf{4. Possible Excitation Energies:}

The excitation energy is:
\[
\Delta E = E_n - E_1 = -54.4\left(\frac{1}{n^2}\right) - (-54.4) = 54.4 \left( 1 - \frac{1}{n^2} \right)
\]

Possible values for $n$:
\begin{itemize}
    \item $n=1$: $\Delta E = 0$ (Elastic collision).
    \item $n=2$: $\Delta E = 54.4(1 - 1/4) = 40.8$ eV.
    \item $n=3$: $\Delta E = 54.4(1 - 1/9) = 48.356$ eV.
    \item $n=4$: $\Delta E = 54.4(1 - 1/16) = 51.0$ eV.
    \item $n=5$: $\Delta E = 54.4(1 - 1/25) = 52.224$ eV.
\end{itemize}

Calculate $K'_n$ for each case:
\begin{itemize}
    \item \textbf{Case n=1 (Elastic):}
    $K'_n = \frac{204.6 - 0}{5} = 40.92$ eV.

    \item \textbf{Case n=2:}
    $K'_n = \frac{204.6 - 4(40.8)}{5} = \frac{204.6 - 163.2}{5} = 8.28$ eV.
    $K_{He} = 0.25(68.2 + 8.28) = 19.12$ eV.

    \item \textbf{Case n=3:}
    $K'_n = \frac{204.6 - 4(48.356)}{5} = \frac{204.6 - 193.424}{5} \approx 2.24$ eV.
    $K_{He} = 0.25(68.2 + 2.24) = 17.61$ eV.

    \item \textbf{Case n=4:}
    $K'_n = \frac{204.6 - 4(51.0)}{5} = \frac{204.6 - 204.0}{5} = 0.12$ eV.
    $K_{He} = 0.25(68.2 + 0.12) = 17.08$ eV.

    \item \textbf{Case n=5:}
    $K'_n = \frac{204.6 - 4(52.224)}{5} < 0$ (Not possible).
\end{itemize}

\vspace{0.5cm}
\fbox{
\begin{minipage}{0.9\textwidth}
\textbf{Answer:} \\
Allowed neutron kinetic energies ($K'_n$): \textbf{8.28 eV, 2.24 eV, 0.12 eV}. \\
Allowed Helium kinetic energies ($K_{He}$): \textbf{19.12 eV, 17.61 eV, 17.08 eV}.
\end{minipage}
}

\newpage
\subsection*{(b) Wavelengths of Emitted Radiation}

\textbf{Problem:}
Shortest and longest wavelengths emitted when $He^+$ de-excites. From part (a), the maximum principal quantum number reached is $n=4$.
The transitions are $n_i \to n_f$.
Photon Energy $E_{\gamma} = E_{n_i} - E_{n_f}$.
Wavelength $\lambda = \frac{hc}{E_{\gamma}} \approx \frac{1240 \text{ eV nm}}{E_{\gamma}}$.

\begin{itemize}
    \item \textbf{Shortest Wavelength ($\lambda_{min}$):}
    Corresponds to maximum energy transition ($n=4 \to n=1$).
    \[
    E_{\gamma, max} = \Delta E_{4 \to 1} = 51.0 \text{ eV}
    \]
    \[
    \lambda_{min} = \frac{1240}{51.0} \approx 24.3 \text{ nm}
    \]

    \item \textbf{Longest Wavelength ($\lambda_{max}$):}
    Corresponds to minimum energy transition between adjacent high-$n$ levels.
    Possibilities: $4 \to 3$, $3 \to 2$, $2 \to 1$.
    \begin{align*}
    \Delta E_{4 \to 3} &= -54.4\left(\frac{1}{16} - \frac{1}{9}\right) \approx 2.64 \text{ eV} \\
    \Delta E_{3 \to 2} &= -54.4\left(\frac{1}{9} - \frac{1}{4}\right) \approx 7.56 \text{ eV} \\
    \Delta E_{2 \to 1} &= 40.8 \text{ eV}
    \end{align*}
    Minimum energy is $2.64$ eV.
    \[
    \lambda_{max} = \frac{1240}{2.64} \approx 470 \text{ nm}
    \]
\end{itemize}

\vspace{0.5cm}
\fbox{
\begin{minipage}{0.9\textwidth}
\textbf{Answer:} \\
Shortest wavelength $\approx$ \textbf{24.3 nm}. \\
Longest wavelength $\approx$ \textbf{470 nm}.
\end{minipage}
}

\vspace{1cm}
\subsection*{(c) Recoil Energy Percentage}

\textbf{Problem:}
Estimate percentage of transition energy $E_0 = E_i - E_f$ that becomes recoil KE for the shortest wavelength case ($4 \to 1$, $E_0 = 51.0$ eV).

\textbf{Solution:}
Conservation of momentum for photon emission: $p_{atom} = p_{\gamma}$.
\[
p_{\gamma} = E_{\gamma}/c \approx E_0/c
\]
Recoil Kinetic Energy:
\[
K_r = \frac{p_{atom}^2}{2M} \approx \frac{(E_0/c)^2}{2M} = \frac{E_0^2}{2Mc^2}
\]

Calculation:
\begin{itemize}
    \item $E_0 = 51.0$ eV.
    \item Mass of $He^+$: $M \approx 4 \times 938$ MeV/$c^2 \approx 3.75 \times 10^9$ eV/$c^2$.
\end{itemize}

Ratio:
\[
\frac{K_r}{E_0} \approx \frac{E_0}{2Mc^2} = \frac{51.0}{2(3.75 \times 10^9)} \approx 6.8 \times 10^{-9}
\]

Percentage:
\[
6.8 \times 10^{-9} \times 100\% = 6.8 \times 10^{-7} \%
\]

\fbox{\textbf{Answer:} $\approx 6.8 \times 10^{-7} \%$.}

\vspace{1cm}
\subsection*{(d) Change in Speed}

Recoil speed:
\[
v_r = \frac{p_{atom}}{M} = \frac{E_0/c}{M} = \frac{E_0}{Mc^2} c
\]
\[
v_r = \frac{51.0}{3.75 \times 10^9} (3.0 \times 10^8) \text{ m/s}
\]
\[
v_r = 1.36 \times 10^{-8} \times 3 \times 10^8 \approx 4.08 \text{ m/s}
\]

\fbox{\textbf{Answer:} $\approx \mathbf{4.1}$ m/s.}

\newpage

% ==========================================
% QUESTION 2
% ==========================================
\section*{Question 2: Falling Particle}
\hrule
\vspace{0.5cm}

\subsection*{(a) Terminal Velocity}

\textbf{Problem:}
Mass $m = 0.1$ kg. Air resistance $F = -kv$, $k = 1.09 \times 10^{-2}$ kg/s. Find terminal velocity $v_T$.

\textbf{Solution:}
At terminal velocity, net force is zero:
\[
mg - k v_T = 0 \implies v_T = \frac{mg}{k}
\]
\[
v_T = \frac{0.1 \times 9.81}{1.09 \times 10^{-2}} = \frac{0.981}{0.0109} = 90.0 \text{ m/s}
\]

\fbox{\textbf{Answer:} $\mathbf{90.0}$ m/s.}

\vspace{1cm}
\subsection*{(b) Time to 99\% Terminal Velocity}

\textbf{Problem:} Time $t$ to reach $v = 0.99 v_T$.

\textbf{Solution:}
Equation of motion:
\[
m \frac{dv}{dt} = mg - kv
\]
Separating variables and integrating:
\begin{align*}
    \int_0^v \frac{dv}{v_T - v} &= \frac{k}{m} \int_0^t dt \\
    -\ln(v_T - v) \Big|_0^v &= \frac{k}{m} t \\
    \ln\left( \frac{v_T}{v_T - v} \right) &= \frac{k}{m} t
\end{align*}
Solving for $t$:
\[
t = \frac{m}{k} \ln\left( \frac{1}{1 - v/v_T} \right)
\]
Substitute $v/v_T = 0.99$:
\[
t = \frac{0.1}{0.0109} \ln\left( \frac{1}{0.01} \right) = 9.174 \times \ln(100)
\]
\[
t = 9.174 \times 4.605 \approx 42.25 \text{ s}
\]

\fbox{\textbf{Answer:} $\mathbf{42.3}$ s.}

\vspace{1cm}
\subsection*{(c) Distance Fallen}

\textbf{Problem:} Distance $y$ when $v = 0.99 v_T$.

\textbf{Solution:}
Velocity as a function of time:
\[
v(t) = v_T (1 - e^{-\frac{k}{m}t})
\]
Integrating to find distance:
\begin{align*}
    y(t) &= \int_0^t v_T (1 - e^{-\frac{k}{m}t'}) dt' \\
    &= v_T \left[ t' + \frac{m}{k} e^{-\frac{k}{m}t'} \right]_0^t \\
    &= v_T \left( t + \frac{m}{k} e^{-\frac{k}{m}t} - \frac{m}{k} \right)
\end{align*}
At the specified time, $e^{-\frac{k}{m}t} = 1 - 0.99 = 0.01$.
\begin{align*}
    y &= 90.0 \left( 42.25 + 9.174(0.01) - 9.174 \right) \\
    &= 90.0 ( 42.25 + 0.0917 - 9.174 ) \\
    &= 90.0 ( 33.168 ) \\
    &\approx 2985 \text{ m}
\end{align*}

\fbox{\textbf{Answer:} $\approx \mathbf{2990}$ m.}

\newpage

% ==========================================
% QUESTION 3
% ==========================================
\section*{Question 3: Chargd particles in B-field}
\hrule
\vspace{0.5cm}

\subsection*{(a) Period of Revolution}
Lorentz force provides centripetal acceleration:
\[
q v B_z = \frac{m v^2}{R} \implies v = \frac{q B_z R}{m}
\]
Angular velocity $\omega = \frac{v}{R} = \frac{q B_z}{m}$.
Period $T = \frac{2\pi}{\omega}$.

\fbox{\textbf{Answer:} $T = \frac{2\pi m}{q B_z}$}

\vspace{1cm}
\subsection*{(b) Acceleration Expressions}
Force $\vec{F} = q (\vec{v} \times \vec{B})$.
\[
\vec{F} = q \begin{vmatrix} \hat{i} & \hat{j} & \hat{k} \\ v_x & v_y & 0 \\ 0 & 0 & B_z \end{vmatrix} = q (v_y B_z \hat{i} - v_x B_z \hat{j})
\]
Acceleration $a = F/m$. Using $\omega = \frac{q B_z}{m} = \frac{2\pi}{T}$:
\begin{align*}
    a_x(t) &= \frac{q B_z}{m} v_y(t) = \frac{2\pi}{T} v_y(t) \\
    a_y(t) &= -\frac{q B_z}{m} v_x(t) = -\frac{2\pi}{T} v_x(t) \\
    a_z(t) &= 0
\end{align*}

\vspace{1cm}
\subsection*{(c) Method 1 (Euler) Equations}
Assume acceleration is constant over $\Delta t$.
\begin{align*}
    v_x(t + \Delta t) &= v_x(t) + \frac{2\pi}{T} v_y(t) \Delta t \\
    v_y(t + \Delta t) &= v_y(t) - \frac{2\pi}{T} v_x(t) \Delta t \\
    x(t + \Delta t) &= x(t) + v_x(t) \Delta t \\
    y(t + \Delta t) &= y(t) + v_y(t) \Delta t
\end{align*}

\vspace{1cm}
\subsection*{(d) Change in Kinetic Energy (Method 1)}
$E_k(t) = \frac{1}{2}m(v_x^2 + v_y^2)$.
Calculate $v_{new}^2 = v_x(t+\Delta t)^2 + v_y(t+\Delta t)^2$:
\begin{align*}
    v_{new}^2 &= \left(v_x + \frac{2\pi \Delta t}{T}v_y\right)^2 + \left(v_y - \frac{2\pi \Delta t}{T}v_x\right)^2 \\
    &= v_x^2 + \left(\frac{2\pi \Delta t}{T}\right)^2 v_y^2 + 2v_x v_y (\dots) + v_y^2 + \left(\frac{2\pi \Delta t}{T}\right)^2 v_x^2 - 2v_y v_x (\dots)
\end{align*}
The cross terms cancel.
\[
v_{new}^2 = (v_x^2 + v_y^2) \left[ 1 + \left(\frac{2\pi \Delta t}{T}\right)^2 \right]
\]
Thus:
\[
\Delta E_k = E_{new} - E_{old} = E_k(t) \left(\frac{2\pi \Delta t}{T}\right)^2
\]

\vspace{1cm}
\subsection*{(e) Ratio of Kinetic Energy after One Orbit}
Number of steps $N = T / \Delta t$.
In each step, energy is multiplied by factor $f = 1 + \left(\frac{2\pi}{N}\right)^2$.
After $N$ steps:
\[
\frac{E_{final}}{E_{initial}} = \left[ 1 + \left(\frac{2\pi}{N}\right)^2 \right]^N
\]
Given $\Delta t = T/100$, so $N=100$.
\[
\text{Ratio} = [1 + (0.0628)^2]^{100} = [1 + 0.00394]^{100} \approx 1.48
\]
\fbox{\textbf{Answer:} $\approx 1.48$}

\vspace{1cm}
\subsection*{(f) Method 2 Equations}
Acceleration is based on velocity which is approximated by central difference:
\[
v_x(t) \approx \frac{x(t+\Delta t) - x(t-\Delta t)}{2\Delta t}
\]
Substituting into $a_x = \frac{2\pi}{T} v_y$ and $a_y = -\frac{2\pi}{T} v_x$:

\begin{align*}
    \frac{x(t+\Delta t) - 2x(t) + x(t-\Delta t)}{(\Delta t)^2} &= \frac{2\pi}{T} \frac{y(t+\Delta t) - y(t-\Delta t)}{2\Delta t} \\
    \frac{y(t+\Delta t) - 2y(t) + y(t-\Delta t)}{(\Delta t)^2} &= -\frac{2\pi}{T} \frac{x(t+\Delta t) - x(t-\Delta t)}{2\Delta t}
\end{align*}

Let $\alpha = \frac{\pi \Delta t}{T}$. The system becomes:
\begin{align*}
    x_{new} - 2x + x_{old} &= \alpha (y_{new} - y_{old}) \\
    y_{new} - 2y + y_{old} &= -\alpha (x_{new} - x_{old})
\end{align*}
This is a system of linear equations for $x_{new}$ and $y_{new}$.

\vspace{1cm}
\subsection*{(g) Change in KE for Method 2}
This method (Central Difference / Leapfrog) is symplectic.
With this specific discretization, the work done $\vec{F} \cdot \vec{v}$ effectively sums to zero over a step because the force is calculated perpendicular to the average velocity.
Thus, $\Delta E_k$ is theoretically zero (or oscillates with bounded error).

\fbox{\textbf{Answer:} $\Delta E_k = 0$.}

\vspace{1cm}
\subsection*{(h) Trajectories}
\begin{itemize}
    \item \textbf{Real:} Circle.
    \item \textbf{Method 1:} Spiral outwards (Energy increases).
    \item \textbf{Method 2:} Closed orbit (Circle/Ellipse) or stable oscillation.
\end{itemize}

\newpage

% ==========================================
% QUESTION 4
% ==========================================
\section*{Question 4: SR: Rocket and Radar}
\hrule
\vspace{0.5cm}

\subsection*{(a) Distance of Rocket at First Reflection}

\textbf{Problem:}
Rocket moves away.
\begin{itemize}
    \item Pulse 1 reflected from Back. Return time $t_{1, ret} = 5.00$ min $= 300$ s.
    \item Pulse 2 reflected from Front. Return time $t_{2, ret} = 300 \text{ s} + 12.0 \text{ \si{\mu s}}$.
\end{itemize}

\textbf{Solution:}
Let Earth be at $x=0$. Rocket starts at $x=0$ at $t=0$.
Time of reflection for Pulse 1:
\[
t_1 = \frac{t_{1, ret}}{2} = 150 \text{ s}
\]
Distance:
\[
d_1 = c t_1 = (3.00 \times 10^8)(150) = 4.50 \times 10^{10} \text{ m}
\]

\fbox{\textbf{Answer:} $4.50 \times 10^{10}$ m.}

\vspace{1cm}
\subsection*{(b) Velocity of Rocket}

\textbf{Solution:}
Reflection 1 (Back): $t_1 = 150$ s, $x_1 = 4.50 \times 10^{10}$ m.
Reflection 2 (Front):
\[
t_2 = \frac{300 + 12 \times 10^{-6}}{2} = 150 + 6 \times 10^{-6} \text{ s}
\]
Distance:
\[
x_2 = c t_2 = c(150) + c(6 \times 10^{-6}) = x_1 + 1800 \text{ m}
\]

Using the Rocket length contraction: $x_F(t) = x_B(t) + L_{contracted}$.
\[
x_1 = v t_1 + x_0 \quad (\text{Back})
\]
\[
x_2 = v t_2 + x_0 + L_0 \sqrt{1 - v^2/c^2} \quad (\text{Front})
\]

Subtracting:
\[
x_2 - x_1 = v(t_2 - t_1) + L_0 \sqrt{1 - v^2/c^2}
\]
\[
1800 = v(6 \times 10^{-6}) + 600 \sqrt{1 - v^2/c^2}
\]

Let $\beta = v/c$. Note that $v(6 \times 10^{-6}) = (c \beta)(6 \times 10^{-6}) = 1800 \beta$.
\[
1800 = 1800 \beta + 600 \sqrt{1-\beta^2}
\]
Divide by 600:
\[
3 = 3\beta + \sqrt{1-\beta^2} \implies 3(1-\beta) = \sqrt{1-\beta^2}
\]
Squaring both sides:
\[
9(1-\beta)^2 = 1-\beta^2 = (1-\beta)(1+\beta)
\]
Divide by $(1-\beta)$ (since $\beta \neq 1$):
\[
9(1-\beta) = 1+\beta \implies 9 - 9\beta = 1 + \beta
\]
\[
8 = 10\beta \implies \beta = 0.8
\]
\[
v = 0.8 c = 2.4 \times 10^8 \text{ m/s}
\]

\fbox{\textbf{Answer:} $0.8 c$.}

\vspace{1cm}
\subsection*{(c) Time Interval in Rocket Frame}

Events: Reflection 1 (Back), Reflection 2 (Front).
Earth frame intervals:
\[
\Delta x = 1800 \text{ m}, \quad \Delta t = 6 \times 10^{-6} \text{ s}
\]

Lorentz Transform to Rocket Frame:
\[
\Delta t' = \gamma \left(\Delta t - \frac{v \Delta x}{c^2}\right)
\]
\[
\gamma = \frac{1}{\sqrt{1-0.8^2}} = \frac{5}{3}
\]

Substitute values:
\[
\Delta t' = \frac{5}{3} \left( 6 \times 10^{-6} - \frac{0.8 c (1800)}{c^2} \right)
\]
\[
\Delta t' = \frac{5}{3} \left( 6 \times 10^{-6} - \frac{0.8 (1800)}{3 \times 10^8} \right)
\]
Note $\frac{1800}{3 \times 10^8} = 6 \times 10^{-6}$.
\[
\Delta t' = \frac{5}{3} \left( 6 \times 10^{-6} - 0.8 (6 \times 10^{-6}) \right)
\]
\[
\Delta t' = \frac{5}{3} (6 \times 10^{-6}) (1 - 0.8) = 10 \times 10^{-6} (0.2)
\]
\[
\Delta t' = 2.0 \times 10^{-6} \text{ s}
\]

\fbox{\textbf{Answer:} $\mathbf{2.0 \text{ \si{\mu s}}}$.}

\vspace{1cm}
\subsection*{(d) Explanation on Time Dilation}
The standard time dilation formula $\Delta t = \gamma \Delta \tau$ applies only when the two events occur at the \textbf{same location} in one of the frames (the proper time frame). In this problem, the two reflections occur at different ends of the rocket (separated by $L_0 = 600$ m in the rocket frame). Since the events are spatially separated in the rocket frame, neither frame measures the proper time interval directly, and the full Lorentz transformation involving both space and time terms must be used.

\end{document}