\documentclass[a4paper,12pt]{article}
\usepackage[utf8]{inputenc}
\usepackage[T1]{fontenc}
\usepackage{amsmath}
\usepackage{amssymb}
\usepackage{graphicx}
\usepackage{geometry}
\usepackage{siunitx}
\usepackage{fancyhdr}
\usepackage{xcolor}

% Page Setup
\geometry{margin=1in}
\pagestyle{fancy}
\fancyhf{}
\lhead{SPhO 2018 Theory Paper 1 Solutions}
\rhead{\thepage}

% Spacing improvements for readability
\setlength{\parskip}{1em}      % Space between paragraphs
\setlength{\parindent}{0pt}    % No indentation for new paragraphs
\linespread{1.15}              % Slightly increased line spacing

\title{\textbf{31\textsuperscript{st} Singapore Physics Olympiad (SPhO) 2018} \\ \Large Theory Paper 1 Solutions}
\author{}
\date{}

\begin{document}

\maketitle
%\tableofcontents
%\newpage

% ==========================================
% QUESTION 1
% ==========================================
\section*{Question 1: Projectile Motion}
\hrule
\vspace{0.5cm}

\subsection*{Problem Statement Summary}
Particle X is projected with speed $V$ at $30^\circ$ from the horizontal. At X's highest point, particle Y is dropped from a tall building. They collide at the base. Y takes $0.17$ s to pass a $5.0$ m window whose base is $50.0$ m above ground.

\textbf{Find:}
\begin{enumerate}
    \item[(a)] Height of the building.
    \item[(b)] Value of $V$.
    \item[(c)] Horizontal distance of X's projection point from the building.
\end{enumerate}

%\newpage
\subsection*{Solution}

\subsubsection*{(a) Height of the Building}

Let the height of the building be $H$.
\begin{itemize}
    \item Window height: $h_w = 5.0$ m.
    \item Base of the window: $h_{base} = 50.0$ m.
    \item Top of the window: $h_{top} = 50.0 + 5.0 = 55.0$ m.
\end{itemize}

Let $t_1$ be the time taken for Y to fall from the roof (height $H$) to the \textbf{top} of the window ($H - 55.0$ m fallen). \\
Let $t_2$ be the time taken for Y to fall from the roof to the \textbf{bottom} of the window ($H - 50.0$ m fallen).

We are given $\Delta t = t_2 - t_1 = 0.17$ s.

Using the kinematic equation $s = \frac{1}{2}gt^2$:
\begin{align*}
    H - 55 &= \frac{1}{2} g t_1^2 \implies t_1 = \sqrt{\frac{2(H-55)}{g}} \\
    H - 50 &= \frac{1}{2} g t_2^2 \implies t_2 = \sqrt{\frac{2(H-50)}{g}}
\end{align*}

Substitute these into the time difference equation $t_2 - t_1 = 0.17$:
\[
\sqrt{\frac{2(H-50)}{9.81}} - \sqrt{\frac{2(H-55)}{9.81}} = 0.17
\]

Let $k = \sqrt{\frac{2}{9.81}} \approx 0.4515$. The equation becomes:
\[
k \left(\sqrt{H-50} - \sqrt{H-55}\right) = 0.17
\]
\[
\sqrt{H-50} - \sqrt{H-55} = \frac{0.17}{0.4515} \approx 0.3765
\]

To solve for $H$, let $A = \sqrt{H-50}$ and $B = \sqrt{H-55}$.
We have the system:
1. $A - B = 0.3765$
2. $A^2 - B^2 = (H-50) - (H-55) = 5$

Using the identity $A^2 - B^2 = (A-B)(A+B)$:
\[
(0.3765)(A+B) = 5 \implies A+B = \frac{5}{0.3765} \approx 13.28
\]

Now, adding the two equations for $A$ and $B$:
\[
(A - B) + (A + B) = 2A \implies 0.3765 + 13.28 = 2A
\]
\[
2A = 13.6565 \implies A \approx 6.828
\]

Finally, solve for $H$:
\[
A^2 = H - 50 = (6.828)^2 \approx 46.62
\]
\[
H = 50 + 46.62 = 96.62 \text{ m}
\]

\vspace{0.5cm}
\fbox{\textbf{Answer:} The height of the building is approximately \textbf{96.6 m}.}

\vspace{1cm}

\subsubsection*{(b) Value of $V$}

Particle Y is dropped when X is at its highest point. They collide at the ground.
First, find the time $t_{fall}$ for Y to fall from height $H$ to the ground:
\[
H = \frac{1}{2} g t_{fall}^2 \implies t_{fall} = \sqrt{\frac{2 \times 96.62}{9.81}} \approx 4.438 \text{ s}
\]

This $t_{fall}$ is also the time it takes for X to travel from its highest point to the ground. For a projectile projected from ground level, the time to reach max height ($t_{up}$) is equal to the time to fall from max height.

Equating $t_{up} = t_{fall} = 4.438$ s:
\[
t_{up} = \frac{V \sin \theta}{g}
\]

Given $\theta = 30^\circ$:
\[
\frac{V \sin 30^\circ}{9.81} = 4.438
\]
\[
V (0.5) = 4.438 \times 9.81 = 43.54
\]
\[
V = 2 \times 43.54 = 87.08 \text{ m/s}
\]

\vspace{0.5cm}
\fbox{\textbf{Answer:} $V \approx \textbf{87.1 m/s}$.}

\vspace{1cm}

\subsubsection*{(c) Distance of Point of Projection}

The horizontal distance $R$ is the range of the projectile. Since the collision occurs at the base, X travels its full horizontal range.

Total flight time $T = 2 \times t_{fall} = 2 \times 4.438 = 8.876$ s.

Horizontal velocity:
\[
V_x = V \cos 30^\circ = 87.08 \times \frac{\sqrt{3}}{2} \approx 75.41 \text{ m/s}
\]

Total distance:
\[
d = V_x \times T = 75.41 \times 8.876 \approx 669 \text{ m}
\]

\vspace{0.5cm}
\fbox{\textbf{Answer:} Distance $\approx \textbf{669 m}$.}

\newpage

% ==========================================
% QUESTION 2
% ==========================================
\section*{Question 2: Simple Harmonic Motion and Friction}
\hrule
\vspace{0.5cm}

\subsection*{(a) Horizontal SHM and Friction}

\textbf{Problem:} Period $T = 2.0$ s. Object slides when amplitude $A = 0.4$ m. Find the coefficient of static friction $\mu_s$.

\textbf{Solution:}

First, calculate the angular frequency:
\[
\omega = \frac{2\pi}{T} = \frac{2\pi}{2.0} = \pi \text{ rad/s}
\]

The maximum acceleration in Simple Harmonic Motion (SHM) is:
\[
a_{max} = \omega^2 A
\]

The force providing this acceleration is static friction. Sliding starts when the required force equals the maximum static friction:
\[
F_{req} = m a_{max} = \mu_s m g
\]

Substituting $a_{max}$:
\[
m(\omega^2 A) = \mu_s m g \implies \mu_s = \frac{\omega^2 A}{g}
\]

Calculation:
\[
\mu_s = \frac{\pi^2 (0.4)}{9.81} = \frac{9.8696 \times 0.4}{9.81} \approx \frac{3.948}{9.81} \approx 0.402
\]

\vspace{0.5cm}
\fbox{\textbf{Answer:} $\mu_s \approx \textbf{0.40}$.}

\vspace{1cm}

\subsection*{(b) Vertical SHM and Contact}

\textbf{Problem:} Period $T' = 1.5$ s. Find maximum amplitude $A'$ for object to remain in contact.

\textbf{Solution:}

Vertical SHM Acceleration: $a(t) = -\omega'^2 y(t)$.
Forces on the object are Gravity $mg$ (down) and Normal force $N$ (up).

Using Newton's 2nd Law (upward positive):
\[
N - mg = m a \implies N = m(g + a)
\]

Contact is lost when $N=0$. This is most critical at the top of the oscillation where acceleration is maximum downwards ($a = -a_{max} = -\omega'^2 A'$).

Condition for contact:
\[
g - \omega'^2 A' \ge 0 \implies A' \le \frac{g}{\omega'^2}
\]

Calculate new angular frequency $\omega'$:
\[
\omega' = \frac{2\pi}{1.5} = \frac{4\pi}{3} \text{ rad/s}
\]

Calculate $A'_{max}$:
\[
A'_{max} = \frac{9.81}{(4\pi/3)^2} = \frac{9.81}{16\pi^2 / 9} = \frac{9.81 \times 9}{16 \times 9.87}
\]
\[
A'_{max} \approx \frac{88.29}{157.9} \approx 0.559 \text{ m}
\]

\vspace{0.5cm}
\fbox{\textbf{Answer:} Max Amplitude $\approx \textbf{0.56 m}$.}

\newpage

% ==========================================
% QUESTION 3
% ==========================================
\section*{Question 3: Atmospheric Physics}
\hrule
\vspace{0.5cm}

\subsection*{(a) Pressure-Height Differential Equation}

Consider a slice of gas of height $\Delta z$ and cross-sectional area $A$.

\begin{itemize}
    \item Mass of slice: $\Delta m = \rho A \Delta z$
    \item Weight of slice: $W = \rho g A \Delta z$
\end{itemize}

Balancing the vertical forces (Pressure up vs Pressure down + Weight):
\[
P(z) A - P(z+\Delta z) A - W = 0
\]
\[
P(z) - P(z+\Delta z) = \rho g \Delta z
\]
\[
\frac{P(z+\Delta z) - P(z)}{\Delta z} = -\rho g
\]

Taking the limit as $\Delta z \to 0$:
\[
\frac{dP}{dz} = -\rho g
\]

\vspace{1cm}

\subsection*{(b) Isothermal Atmosphere Pressure Law}

Assume the Ideal Gas Law applies:
\[
PV = NkT \implies P = \frac{N}{V} k T = n k T
\]
where $n$ is the number density.

The density $\rho$ is given by $\rho = n m_{molecule}$, so:
\[
\rho = \frac{P m_{molecule}}{k T}
\]

Substitute this into the differential equation derived in part (a):
\[
\frac{dP}{dz} = - \left( \frac{P m g}{k T} \right)
\]

Rearrange to separate variables:
\[
\frac{dP}{P} = - \frac{mg}{kT} dz
\]

Integrate from $z=0$ (where pressure is $P(0)$) to height $z$:
\[
\int_{P(0)}^{P(z)} \frac{dP}{P} = \int_{0}^{z} - \frac{mg}{kT} dz
\]
\[
\ln \frac{P(z)}{P(0)} = - \frac{mg}{kT} z
\]

Exponentiating both sides gives the final expression:
\[
P(z) = P(0) e^{-\frac{mgz}{kT}}
\]

\newpage

% ==========================================
% QUESTION 4
% ==========================================
\section*{Question 4: Waves on Strings}
\hrule
\vspace{0.5cm}

\subsection*{(a) Wave Speed Equation}

The speed $v$ of a wave on a string depends on the tension $F_T$ and the linear mass density $\mu = m/l$.

\vspace{0.5cm}
\fbox{Equation: \quad $v = \sqrt{\frac{F_T}{\mu}} = \sqrt{\frac{F_T l}{m}}$}

\vspace{1cm}

\subsection*{(b) Composition of Brass Wire}

\textbf{Given:}
\begin{itemize}
    \item Copper wire frequency: $f_{Cu} = 256 \pm 5$ Hz (Beat frequency is 5 Hz).
    \item Brass wire frequency: $f_{Brass} = 256$ Hz (Resonance).
    \item Physical dimensions ($l, A$) and Tension ($F_T$) are the same for both wires.
    \item Densities: $\rho_{Cu} = 8940 \text{ kg/m}^3$, $\rho_{Zn} = 7140 \text{ kg/m}^3$.
\end{itemize}

\textbf{1. Determine Frequency Relationship:}

The fundamental frequency is given by:
\[
f = \frac{v}{2l} = \frac{1}{2l} \sqrt{\frac{F_T}{\rho A}}
\]
Since $l, F_T, A$ are constant, $f$ is inversely proportional to the square root of density:
\[
f \propto \frac{1}{\sqrt{\rho}} \implies \frac{f_{Cu}}{f_{Brass}} = \sqrt{\frac{\rho_{Brass}}{\rho_{Cu}}}
\]
Squaring both sides:
\[
\rho_{Brass} = \rho_{Cu} \left( \frac{f_{Cu}}{f_{Brass}} \right)^2
\]

\textbf{2. Determine $f_{Cu}$:}

Since $\rho_{Zn} < \rho_{Cu}$, adding Zinc to Copper to make Brass decreases the density ($\rho_{Brass} < \rho_{Cu}$).
Lower density implies higher frequency. Thus, we expect $f_{Brass} > f_{Cu}$.
Given $f_{Brass} = 256$ Hz, and the beat frequency is 5 Hz, $f_{Cu}$ must be $256 - 5 = 251$ Hz.

\textbf{3. Calculate Brass Density:}

\[
\rho_{Brass} = 8940 \left( \frac{251}{256} \right)^2 = 8940 (0.9612) \approx 8593 \text{ kg/m}^3
\]

\textbf{4. Calculate Zinc Composition:}

Let $x$ be the percentage by mass of Zinc.
Assumption: The total volume is the sum of the constituent volumes ($V_{total} = V_{Cu} + V_{Zn}$).

The density of the alloy is:
\[
\frac{1}{\rho_{Brass}} = \frac{V_{total}}{M_{total}} = \frac{V_{Zn} + V_{Cu}}{M_{total}}
\]

Using $V = M/\rho$:
\[
\frac{1}{\rho_{Brass}} = \frac{x}{\rho_{Zn}} + \frac{1-x}{\rho_{Cu}}
\]

Substitute the values:
\[
\frac{1}{8593} = \frac{x}{7140} + \frac{1-x}{8940}
\]

To solve for $x$, multiply by $10^8$:
\[
\frac{10^8}{8593} \approx 11637, \quad \frac{10^8}{7140} \approx 14006, \quad \frac{10^8}{8940} \approx 11186
\]
\[
11637 = 14006 x + 11186 (1-x)
\]
\[
11637 = 14006 x + 11186 - 11186 x
\]
\[
11637 - 11186 = (14006 - 11186) x
\]
\[
451 = 2820 x \implies x = \frac{451}{2820} \approx 0.160
\]

\vspace{0.5cm}
\fbox{\textbf{Answer:} Zinc percentage by mass is approximately \textbf{16\%}.}

\newpage

% ==========================================
% QUESTION 5
% ==========================================
\section*{Question 5: Radioactive Decay and Blood Volume}
\hrule
\vspace{0.5cm}

\subsection*{(a) Decay Constant}

Given Half-life $T_{1/2} = 8.025$ days.
\[
\lambda = \frac{\ln 2}{T_{1/2}} = \frac{0.6931}{8.025} \approx 0.08637 \text{ day}^{-1}
\]

\vspace{1cm}

\subsection*{(b) Total Blood Volume}

\textbf{Given:}
\begin{itemize}
    \item Injected volume $V_{inj} = \SI{5.00}{ml}$.
    \item Concentration of ${}^{131}I$: $C_0 = 1.00 \times 10^{-10} \text{ kg/m}^3$.
    \item Sample volume $V_{sample} = \SI{5.00}{ml}$.
    \item Time elapsed $t = 24$ hours = 1 day.
    \item Sample Activity: $N_{counts} = 3171$ in 30 mins.
\end{itemize}

\textbf{1. Calculate Initial Amount of Iodine:}
\[
m_0 = C_0 V_{inj} = (1.00 \times 10^{-10} \text{ kg/m}^3) (5.00 \times 10^{-6} \text{ m}^3) = 5.00 \times 10^{-16} \text{ kg}
\]

Calculate number of atoms $N_0$. Molar mass of ${}^{131}I \approx 0.131$ kg/mol.
\[
N_0 = \frac{5.00 \times 10^{-16}}{0.131} \times 6.02 \times 10^{23} \approx 2.30 \times 10^9 \text{ atoms}
\]

\textbf{2. Amount Remaining after 1 day:}
Using radioactive decay law $N(t) = N_0 e^{-\lambda t}$ with $t=1$ day:
\[
N(1) = 2.30 \times 10^9 \times e^{-0.08637} = 2.30 \times 10^9 \times 0.917 = 2.11 \times 10^9 \text{ atoms}
\]
This total number of atoms is distributed throughout the total blood volume $V_{total}$.

\textbf{3. Activity of Sample:}
Measured Rate:
\[
R_{sample} = \frac{3171 \text{ counts}}{1800 \text{ s}} \approx 1.76 \text{ Bq}
\]
Activity is given by $A = \lambda_{sec} N_{sample}$.
First, convert $\lambda$ to seconds:
\[
\lambda_{sec} = \frac{0.08637}{24 \times 3600} \approx 1.00 \times 10^{-6} \text{ s}^{-1}
\]
Now find atoms in the sample ($N_{sample}$):
\[
N_{sample} = \frac{R_{sample}}{\lambda_{sec}} = \frac{1.76}{1.00 \times 10^{-6}} \approx 1.76 \times 10^6 \text{ atoms}
\]

\textbf{4. Volume Calculation:}
Assuming concentration is uniform, the ratio of atoms equals the ratio of volumes:
\[
\frac{N_{sample}}{V_{sample}} = \frac{N(1)}{V_{total}}
\]
\[
V_{total} = V_{sample} \times \frac{N(1)}{N_{sample}}
\]
\[
V_{total} = 5.00 \text{ ml} \times \frac{2.11 \times 10^9}{1.76 \times 10^6}
\]
\[
V_{total} = 5.00 \times 1200 = 6000 \text{ ml} = 6.00 \text{ Liters}
\]

\vspace{0.5cm}
\fbox{\textbf{Answer:} Total blood volume $\approx \textbf{6.00 L}$.}

\end{document}