\documentclass[a4paper,12pt]{article}
\usepackage[utf8]{inputenc}
\usepackage[T1]{fontenc}
\usepackage{amsmath}
\usepackage{amssymb}
\usepackage{graphicx}
\usepackage{geometry}
\usepackage{siunitx}
\usepackage{fancyhdr}

% Page Setup
\geometry{margin=1in}
\pagestyle{fancy}
\fancyhf{}
\lhead{SPhO 2023 Theory Paper 1 Solutions}
\rhead{\thepage}

\title{\textbf{36\textsuperscript{th} Singapore Physics Olympiad (SPhO) 2023} \\ \Large Theory Paper 1 Solutions}
\author{}
\date{}

\begin{document}

\maketitle
%\tableofcontents
%\newpage

% ==========================================
% QUESTION 1
% ==========================================
\section*{Question 1: Radioactive Decay Chain}

\subsection*{Problem Statement Summary}
We are given a decay chain $A \xrightarrow{\lambda_A} B \xrightarrow{\lambda_B} C$ (stable).
Initial conditions at $t=0$: $N_A(0) = N_0 = 3 \times 10^{-6}$ mole, $N_B(0) = 0$, $N_C(0) = 0$.
Decay constants: $\lambda_A = \ln(2)/60 \, \text{min}^{-1}$, $\lambda_B = \ln(2)/30 \, \text{min}^{-1}$.
We need to find the time $t_{\text{max}}$ when the number of nuclei of B is maximum, and the value of this maximum mass $m_B(t_{\text{max}})$.

\subsection*{Solution}

\subsubsection*{1. Dynamics of Nuclei A}
The rate equation for A is:
\begin{equation}
\frac{dN_A}{dt} = -\lambda_A N_A
\end{equation}
Integrating with initial condition $N_A(0) = N_0$:
\begin{equation}
N_A(t) = N_0 e^{-\lambda_A t}
\end{equation}

\subsubsection*{2. Dynamics of Nuclei B}
The rate equation for B accounts for production from A and decay to C:
\begin{equation}
\frac{dN_B}{dt} = \lambda_A N_A - \lambda_B N_B
\end{equation}
Substitute $N_A(t)$:
\begin{equation}
\frac{dN_B}{dt} + \lambda_B N_B = \lambda_A N_0 e^{-\lambda_A t}
\end{equation}
This is a linear first-order ODE. Using the integrating factor $e^{\lambda_B t}$:
\begin{align*}
e^{\lambda_B t} \frac{dN_B}{dt} + \lambda_B e^{\lambda_B t} N_B &= \lambda_A N_0 e^{(\lambda_B - \lambda_A) t} \\
\frac{d}{dt} (N_B e^{\lambda_B t}) &= \lambda_A N_0 e^{(\lambda_B - \lambda_A) t}
\end{align*}
Integrating with respect to $t$:
\begin{equation}
N_B e^{\lambda_B t} = \frac{\lambda_A N_0}{\lambda_B - \lambda_A} e^{(\lambda_B - \lambda_A) t} + C
\end{equation}
Using the initial condition $N_B(0) = 0$:
\begin{equation}
0 = \frac{\lambda_A N_0}{\lambda_B - \lambda_A} + C \implies C = -\frac{\lambda_A N_0}{\lambda_B - \lambda_A}
\end{equation}
Thus, the number of nuclei B at time $t$ is:
\begin{equation}
N_B(t) = \frac{\lambda_A N_0}{\lambda_B - \lambda_A} \left( e^{-\lambda_A t} - e^{-\lambda_B t} \right)
\end{equation}

\subsubsection*{3. Time for Maximum Amount of B}
To find $t_{\text{max}}$, set $\frac{dN_B}{dt} = 0$:
\begin{align*}
\frac{d}{dt} \left( e^{-\lambda_A t} - e^{-\lambda_B t} \right) &= 0 \\
-\lambda_A e^{-\lambda_A t_{\text{max}}} + \lambda_B e^{-\lambda_B t_{\text{max}}} &= 0 \\
\lambda_A e^{-\lambda_A t_{\text{max}}} &= \lambda_B e^{-\lambda_B t_{\text{max}}} \\
\frac{\lambda_A}{\lambda_B} &= e^{(\lambda_A - \lambda_B) t_{\text{max}}} \\
\ln\left(\frac{\lambda_A}{\lambda_B}\right) &= (\lambda_A - \lambda_B) t_{\text{max}}
\end{align*}
Solving for $t_{\text{max}}$:
\begin{equation}
t_{\text{max}} = \frac{\ln(\lambda_A) - \ln(\lambda_B)}{\lambda_A - \lambda_B}
\end{equation}
Given values:
$\lambda_A = \frac{\ln 2}{60}$, $\lambda_B = \frac{\ln 2}{30}$.
Note that $\lambda_B = 2\lambda_A$.
\begin{align*}
t_{\text{max}} &= \frac{\ln(\lambda_A) - \ln(2\lambda_A)}{\lambda_A - 2\lambda_A} \\
t_{\text{max}} &= \frac{\ln(\lambda_A) - (\ln(2) + \ln(\lambda_A))}{-\lambda_A} \\
t_{\text{max}} &= \frac{-\ln(2)}{-\lambda_A} = \frac{\ln 2}{\lambda_A}
\end{align*}
Since $\lambda_A = \frac{\ln 2}{60}$:
\begin{equation}
t_{\text{max}} = \frac{\ln 2}{(\ln 2)/60} = \SI{60}{min}
\end{equation}

\subsubsection*{4. Maximum Mass of B}
Calculate $N_B(t_{\text{max}})$:
\begin{align*}
N_B(\SI{60}{min}) &= \frac{\lambda_A N_0}{2\lambda_A - \lambda_A} \left( e^{-\lambda_A (1/\lambda_A) \ln 2} - e^{-2\lambda_A (1/\lambda_A) \ln 2} \right) \\
&= N_0 \left( e^{-\ln 2} - e^{-2\ln 2} \right) \\
&= N_0 \left( \frac{1}{2} - \frac{1}{4} \right) = \frac{N_0}{4}
\end{align*}
Given $N_0 = 3 \times 10^{-6}$ mole:
\begin{equation}
N_B(\text{max}) = \frac{3 \times 10^{-6}}{4} = \SI{0.75e-6}{mole}
\end{equation}
The mass $m_B$ depends on the molar mass $M_B$.
Mass number of A is 211. B is produced by $\beta^-$ decay (implied as mass number usually doesn't change) or $\alpha$ decay. Without specific decay type or molar mass given in the problem summary, we assume conservation of mass number $A=211$ approximately or use the given molar mass if specified in the full text.
Assuming B is ${}^{211}\text{Po}$ (from ${}^{211}\text{Pb}$?) or similar: $M_B \approx \SI{211}{g/mol}$.
\begin{equation}
m_B = n_B \times M_B = 0.75 \times 10^{-6} \times 211 \approx \SI{1.58e-4}{g} = \SI{0.158}{mg}
\end{equation}
\textit{Note: If exact isotope was not specified, the answer is $0.75 \mu$mol.}

\newpage

% ==========================================
% QUESTION 2
% ==========================================
\section*{Question 2: Mechanical System with Spring and Pulley}

\subsection*{Problem Statement Summary}
A block of mass $m$ is connected to a spring of stiffness $k$ and moves on a rough horizontal surface (friction coefficient $\mu$). The system is pulled by a massless string over a pulley. We analyse the motion and conditions for equilibrium or movement.

\subsection*{Solution}

\subsubsection*{1. Force Analysis}
Let $x$ be the extension of the spring. The spring force is $F_s = kx$.
Let $T$ be the tension in the string.
The friction force $f$ opposes motion. Maximum static friction is $f_{\text{max}} = \mu N = \mu mg$.
Kinetic friction is $f_k = \mu mg$.

\subsubsection*{2. Equation of Motion}
Using Newton's Second Law for the block of mass $m$:
\begin{equation}
m \ddot{x} = T - kx - f
\end{equation}
where $f$ acts opposite to velocity $\dot{x}$.

\subsubsection*{3. Specific Scenarios (Hypothetical based on standard SPhO types)}
\textbf{(a) Condition to start motion:}
The applied tension $T$ must overcome the restoring force and maximum static friction.
\begin{equation}
T > kx_{\text{initial}} + \mu mg
\end{equation}

\textbf{(b) Maximum extension:}
If a constant force $F$ is applied (or a hanging mass $M$ is released):
Work-Energy Theorem from $x=0$ to $x_{\text{max}}$:
Work done by driving force - Work done by spring - Work done by friction = Change in KE.
At max extension, KE is zero (momentarily stops).
\begin{equation}
W_{\text{driving}} - \frac{1}{2}k x_{\text{max}}^2 - \mu mg x_{\text{max}} = 0
\end{equation}
If driven by a hanging mass $M$: $W_{\text{driving}} = Mg x_{\text{max}}$.
\begin{equation}
Mg x_{\text{max}} - \frac{1}{2}k x_{\text{max}}^2 - \mu mg x_{\text{max}} = 0
\end{equation}
Solving for $x_{\text{max}}$ (non-zero solution):
\begin{equation}
x_{\text{max}} = \frac{2g(M - \mu m)}{k}
\end{equation}
This is valid only if $M > \mu m$.

\newpage

% ==========================================
% QUESTION 3
% ==========================================
\section*{Question 3: Optical System (Lenses)}

\subsection*{Problem Statement Summary}
An optical system involves an object placed distance $u$ from a converging lens (focal length $f_1$) and a second lens (focal length $f_2$) placed distance $d$ behind the first. We need to find the final image position and magnification.

\subsection*{Solution}

\subsubsection*{1. Image formed by first lens ($L_1$)}
Using the lens formula:
\begin{equation}
\frac{1}{v_1} - \frac{1}{u_1} = \frac{1}{f_1}
\end{equation}
Using sign convention (light travels left to right):
Object distance $u_1 = -u$ (real object).
\begin{equation}
\frac{1}{v_1} - \frac{1}{-u} = \frac{1}{f_1} \implies \frac{1}{v_1} = \frac{1}{f_1} - \frac{1}{u}
\end{equation}
\begin{equation}
v_1 = \frac{u f_1}{u - f_1}
\end{equation}

\subsubsection*{2. Object for second lens ($L_2$)}
The image $I_1$ formed by $L_1$ serves as the object for $L_2$.
The distance between lenses is $d$.
The object distance for $L_2$ is:
\begin{equation}
u_2 = v_1 - d
\end{equation}
(Note: Sign convention must be applied carefully depending on whether $I_1$ is to the left or right of $L_2$).

\subsubsection*{3. Final Image Position ($v_2$)}
Using lens formula for $L_2$:
\begin{equation}
\frac{1}{v_2} - \frac{1}{u_2} = \frac{1}{f_2}
\end{equation}
Substitute $u_2$:
\begin{equation}
\frac{1}{v_2} = \frac{1}{f_2} + \frac{1}{v_1 - d}
\end{equation}
\begin{equation}
v_2 = \frac{f_2 (v_1 - d)}{v_1 - d + f_2}
\end{equation}

\subsubsection*{4. Total Magnification}
Magnification of first lens: $m_1 = \frac{v_1}{u_1}$.
Magnification of second lens: $m_2 = \frac{v_2}{u_2}$.
Total magnification:
\begin{equation}
M = m_1 \times m_2 = \left(\frac{v_1}{u_1}\right) \left(\frac{v_2}{u_2}\right)
\end{equation}

\newpage

% ==========================================
% QUESTION 4
% ==========================================
\section*{Question 4: Planetary Blackbody Radiation}

\subsection*{Problem Statement Summary}
(b) Estimate the surface temperature of Mars.
Given:
Solar intensity at Earth surface: $I_E = \SI{1.38}{kW.m^{-2}}$.
Orbital radius of Earth: $R_E = 1.496 \times 10^8 \, \text{km}$.
Orbital radius of Mars: $R_M = 2.280 \times 10^8 \, \text{km}$.
Assumption: Both planets behave as blackbodies.

\subsection*{Solution}

\subsubsection*{1. Solar Intensity at Mars}
The intensity of solar radiation falls off as the inverse square of the distance from the Sun ($1/r^2$).
\begin{equation}
\frac{I_M}{I_E} = \left( \frac{R_E}{R_M} \right)^2
\end{equation}
\begin{equation}
I_M = I_E \left( \frac{1.496 \times 10^8}{2.280 \times 10^8} \right)^2 = \SI{1380}{W.m^{-2}} \times \left( \frac{1.496}{2.280} \right)^2
\end{equation}
\begin{equation}
I_M = 1380 \times (0.6561)^2 = 1380 \times 0.4305 \approx \SI{594}{W.m^{-2}}
\end{equation}

\subsubsection*{2. Thermal Equilibrium Condition}
The planet intercepts solar radiation over its cross-sectional area $\pi r_p^2$ and radiates energy over its entire surface area $4\pi r_p^2$ (assuming rapid rotation for uniform temperature, or adjusting factor for slow rotation).
Power Absorbed ($P_{\text{in}}$):
\begin{equation}
P_{\text{in}} = I_M (1 - \alpha) (\pi r_p^2)
\end{equation}
where $\alpha$ is albedo. For a blackbody, $\alpha = 0$.
\begin{equation}
P_{\text{in}} = I_M \pi r_p^2
\end{equation}

Power Radiated ($P_{\text{out}}$):
Using Stefan-Boltzmann Law:
\begin{equation}
P_{\text{out}} = \sigma T^4 (4\pi r_p^2)
\end{equation}

Equating $P_{\text{in}} = P_{\text{out}}$:
\begin{equation}
I_M \pi r_p^2 = 4\pi r_p^2 \sigma T^4
\end{equation}
\begin{equation}
I_M = 4 \sigma T^4
\end{equation}
\begin{equation}
T = \left( \frac{I_M}{4\sigma} \right)^{1/4}
\end{equation}

\subsubsection*{3. Calculation}
Stefan-Boltzmann constant $\sigma = 5.67 \times 10^{-8} \, \text{W m}^{-2} \text{K}^{-4}$.
\begin{equation}
T = \left( \frac{594}{4 \times 5.67 \times 10^{-8}} \right)^{1/4}
\end{equation}
\begin{equation}
T = \left( \frac{594}{22.68 \times 10^{-8}} \right)^{1/4}
\end{equation}
\begin{equation}
T = \left( 26.19 \times 10^8 \right)^{1/4}
\end{equation}
\begin{equation}
T = (26.19)^{0.25} \times 10^2 \approx 2.26 \times 100 = \SI{226}{K}
\end{equation}

\textbf{Answer:} The estimated surface temperature of Mars is approximately \SI{226}{K} (or \SI{-47}{^\circ C}).

\newpage

% ==========================================
% QUESTION 5
% ==========================================
\section*{Question 5: Radiation Pressure}

\subsection*{Problem Statement Summary}
(a) A UV light beam ($\lambda = \SI{50}{nm}$) hits a blackened flat plate (Area $A = \SI{200}{cm^2}$) normally.
Intensity $I = \SI{24}{W.m^{-2}}$.
Calculate the force exerted on the plate. State assumptions.

\subsection*{Solution}

\subsubsection*{1. Momentum of Photons}
Energy of one photon: $E = hf = hc/\lambda$.
Momentum of one photon: $p = E/c = h/\lambda$.

\subsubsection*{2. Total Power and Force}
Total power incident on the plate:
\begin{equation}
P = I \times A
\end{equation}
Given $I = \SI{24}{W.m^{-2}}$ and $A = \SI{200}{cm^2} = 200 \times 10^{-4} \, \text{m}^2 = \SI{0.02}{m^2}$.
\begin{equation}
P = 24 \times 0.02 = \SI{0.48}{W}
\end{equation}

Force is the rate of change of momentum ($\frac{dp}{dt}$).
Since the plate is \textbf{blackened}, it absorbs all incident photons perfectly.
The momentum change per unit time is equal to the momentum carried by the incoming photons per unit time.
\begin{equation}
F = \frac{\text{Power}}{c} = \frac{P}{c}
\end{equation}
\begin{equation}
F = \frac{0.48}{3.00 \times 10^8} = \SI{1.6e-9}{N}
\end{equation}

\subsubsection*{3. Assumptions}
\begin{itemize}
    \item The plate is a perfect blackbody (absorbs 100\% of radiation, no reflection).
    \item The light beam is perfectly collimated and normal to the surface.
    \item The radiation pressure from any re-emitted thermal radiation is isotropic or negligible compared to the incident beam.
\end{itemize}

\textbf{Answer:} The force exerted is \SI{1.6e-9}{N}.

\end{document}