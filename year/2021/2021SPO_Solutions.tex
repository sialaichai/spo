\documentclass[a4paper,12pt]{article}
\usepackage[utf8]{inputenc}
\usepackage[T1]{fontenc}
\usepackage{amsmath}
\usepackage{amssymb}
\usepackage{graphicx}
\usepackage{geometry}
\usepackage{siunitx}
\usepackage{fancyhdr}
\usepackage{tikz}

% Page Setup
\geometry{margin=1in}
\pagestyle{fancy}
\fancyhf{}
\lhead{SPhO 2021 Theory Solutions}
\rhead{\thepage}

\title{\textbf{34\textsuperscript{th} Singapore Physics Olympiad (SPhO) 2021} \\ \Large Theory Paper Solutions}
\author{}
\date{}

\begin{document}

\maketitle
%\tableofcontents
%\newpage

% ==========================================
% QUESTION 1
% ==========================================
\section*{Question 1: Composite Capacitor}

\subsection*{Problem Statement Summary}
A parallel-plate capacitor has plate area $A = 10.5 \text{ cm}^2$ and separation $2d = 7.12 \text{ mm}$. The gap is filled with three dielectrics:
\begin{itemize}
    \item Left half (area $A/2$, thickness $2d$): $\kappa_1 = 21.0$.
    \item Top Right (area $A/2$, thickness $d$): $\kappa_2 = 42.0$.
    \item Bottom Right (area $A/2$, thickness $d$): $\kappa_3 = 58.0$.
\end{itemize}
Calculate the total capacitance.

\subsection*{Solution}

The system can be modeled as a combination of three ideal capacitors.
\begin{enumerate}
    \item **Capacitor 1 ($C_1$):** Corresponds to the left half.
    \begin{itemize}
        \item Plate area: $A_1 = A/2$
        \item Plate separation: $d_1 = 2d$
        \item Dielectric constant: $\kappa_1 = 21.0$
    \end{itemize}
    \[ C_1 = \frac{\kappa_1 \epsilon_0 A_1}{d_1} = \frac{\kappa_1 \epsilon_0 (A/2)}{2d} = \frac{\kappa_1 \epsilon_0 A}{4d} \]

    \item **Capacitor 2 ($C_2$):** Corresponds to the top right section.
    \begin{itemize}
        \item Plate area: $A_2 = A/2$
        \item Plate separation: $d_2 = d$
        \item Dielectric constant: $\kappa_2 = 42.0$
    \end{itemize}
    \[ C_2 = \frac{\kappa_2 \epsilon_0 A_2}{d_2} = \frac{\kappa_2 \epsilon_0 (A/2)}{d} = \frac{\kappa_2 \epsilon_0 A}{2d} \]

    \item **Capacitor 3 ($C_3$):** Corresponds to the bottom right section.
    \begin{itemize}
        \item Plate area: $A_3 = A/2$
        \item Plate separation: $d_3 = d$
        \item Dielectric constant: $\kappa_3 = 58.0$
    \end{itemize}
    \[ C_3 = \frac{\kappa_3 \epsilon_0 A_3}{d_3} = \frac{\kappa_3 \epsilon_0 (A/2)}{d} = \frac{\kappa_3 \epsilon_0 A}{2d} \]
\end{enumerate}

**Equivalent Circuit:**
- $C_2$ and $C_3$ are stacked vertically in the gap, so they are in **series**. The effective capacitance of the right half, $C_R$, is:
\[ \frac{1}{C_R} = \frac{1}{C_2} + \frac{1}{C_3} \]
\[ \frac{1}{C_R} = \frac{2d}{\kappa_2 \epsilon_0 A} + \frac{2d}{\kappa_3 \epsilon_0 A} = \frac{2d}{\epsilon_0 A} \left( \frac{1}{\kappa_2} + \frac{1}{\kappa_3} \right) \]
\[ C_R = \frac{\epsilon_0 A}{2d} \left( \frac{\kappa_2 \kappa_3}{\kappa_2 + \kappa_3} \right) \]

- The left capacitor $C_1$ and the equivalent right capacitor $C_R$ are side-by-side, so they are in **parallel**.
\[ C_{total} = C_1 + C_R \]
\[ C_{total} = \frac{\kappa_1 \epsilon_0 A}{4d} + \frac{\epsilon_0 A}{2d} \left( \frac{\kappa_2 \kappa_3}{\kappa_2 + \kappa_3} \right) = \frac{\epsilon_0 A}{2d} \left( \frac{\kappa_1}{2} + \frac{\kappa_2 \kappa_3}{\kappa_2 + \kappa_3} \right) \]

**Calculation:**
Given values:
$A = 10.5 \text{ cm}^2 = 1.05 \times 10^{-3} \text{ m}^2$
$2d = 7.12 \text{ mm} \implies d = 3.56 \text{ mm} = 3.56 \times 10^{-3} \text{ m}$
$\epsilon_0 = 8.85 \times 10^{-12} \text{ F/m}$

First, calculate the geometric factor:
\[ \frac{\epsilon_0 A}{2d} = \frac{(8.85 \times 10^{-12})(1.05 \times 10^{-3})}{7.12 \times 10^{-3}} = \frac{9.2925 \times 10^{-15}}{7.12 \times 10^{-3}} \approx 1.305 \times 10^{-12} \text{ F} = 1.305 \text{ pF} \]

Next, calculate the dielectric term:
\[ \text{Term 1} = \frac{\kappa_1}{2} = \frac{21.0}{2} = 10.5 \]
\[ \text{Term 2} = \frac{\kappa_2 \kappa_3}{\kappa_2 + \kappa_3} = \frac{42.0 \times 58.0}{42.0 + 58.0} = \frac{2436}{100} = 24.36 \]
\[ \text{Total Dielectric Factor} = 10.5 + 24.36 = 34.86 \]

Total Capacitance:
\[ C_{total} = (1.305 \text{ pF}) \times 34.86 \approx 45.49 \text{ pF} \]

\textbf{Answer:} The capacitance is approximately \textbf{45.5 pF}.

\newpage

% ==========================================
% QUESTION 2
% ==========================================
\section*{Question 2: Electrostatics-Potential}

\subsection*{Problem Statement Summary}
Rectangle with sides 5.0 cm and 15.0 cm.
Charges: $q_1 = -5.0 \mu\text{C}$ (Top-Left), $q_2 = +2.0 \mu\text{C}$ (Bottom-Right).
Points: A (Bottom-Left), B (Top-Right).

\subsection*{Solution}

Distances:
- Side 1 (Vertical): $h = 5.0 \text{ cm} = 0.05 \text{ m}$.
- Side 2 (Horizontal): $w = 15.0 \text{ cm} = 0.15 \text{ m}$.
- Diagonal: $r_{diag} = \sqrt{0.05^2 + 0.15^2} = \sqrt{0.0025 + 0.0225} = \sqrt{0.025} \approx 0.1581 \text{ m}$.

Coulomb Constant $k = \frac{1}{4\pi\epsilon_0} \approx 8.99 \times 10^9 \text{ N m}^2\text{C}^{-2}$.

\subsubsection*{(a) Electric Potential at corner A}
A is at the bottom-left.
- Distance to $q_1$ (top-left): $r_{1A} = 0.05 \text{ m}$.
- Distance to $q_2$ (bottom-right): $r_{2A} = 0.15 \text{ m}$.
\[ V_A = k \left( \frac{q_1}{r_{1A}} + \frac{q_2}{r_{2A}} \right) \]
\[ V_A = 8.99 \times 10^9 \left( \frac{-5.0 \times 10^{-6}}{0.05} + \frac{2.0 \times 10^{-6}}{0.15} \right) \]
\[ V_A = 8.99 \times 10^3 \left( -100 + 13.33 \right) = 8.99 \times 10^3 (-86.67) \approx -7.79 \times 10^5 \text{ V} \]

\subsubsection*{(b) Electric Potential at corner B}
B is at the top-right.
- Distance to $q_1$ (top-left): $r_{1B} = 0.15 \text{ m}$.
- Distance to $q_2$ (bottom-right): $r_{2B} = 0.05 \text{ m}$.
\[ V_B = k \left( \frac{q_1}{r_{1B}} + \frac{q_2}{r_{2B}} \right) \]
\[ V_B = 8.99 \times 10^9 \left( \frac{-5.0 \times 10^{-6}}{0.15} + \frac{2.0 \times 10^{-6}}{0.05} \right) \]
\[ V_B = 8.99 \times 10^3 \left( -33.33 + 40.0 \right) = 8.99 \times 10^3 (6.67) \approx +6.00 \times 10^4 \text{ V} \]

\subsubsection*{(c) Work required to move $q_3 = +3.0 \mu\text{C}$ from B to A}
Work done by external agent:
\[ W_{ext} = q_3 (V_A - V_B) \]
\[ W_{ext} = (3.0 \times 10^{-6}) (-7.79 \times 10^5 - 0.60 \times 10^5) \]
\[ W_{ext} = 3.0 \times 10^{-6} (-8.39 \times 10^5) = -2.52 \text{ J} \]
(Note: Negative work means energy is released or work is done *by* the field).

\subsubsection*{(d) Change in Potential Energy}
The work done by the external agent is equal to the change in potential energy of the system.
\[ \Delta U = W_{ext} = -2.52 \text{ J} \]
Since $\Delta U$ is negative, the potential energy **decreases**.

\subsubsection*{(e) \& (f) Path Dependence}
The electrostatic force is conservative. The work done depends only on the initial and final positions, not on the path taken.
(e) Inside rectangle: **Same work**.
(f) Outside rectangle: **Same work**.

\newpage

% ==========================================
% QUESTION 3
% ==========================================
\section*{Question 3: EMI-Motional EMF}

\subsection*{Problem Statement Summary}
Rod length $L = 10 \text{ cm} = 0.1 \text{ m}$. Velocity $v = 5.0 \text{ m/s}$ (to the right). Magnetic field $B = 1.2 \text{ T}$ (out of page). Resistance $R = 0.40 \, \Omega$.

\subsection*{Solution}

\subsubsection*{(a) Magnitude of EMF}
\[ \mathcal{E} = B L v = (1.2)(0.1)(5.0) = 0.60 \text{ V} \]

\subsubsection*{(b) Direction of EMF}
Using the magnetic force law $\vec{F} = q(\vec{v} \times \vec{B})$.
$\vec{v}$ is to the right ($+x$), $\vec{B}$ is out of the page ($+z$).
$\vec{v} \times \vec{B}$ is in the direction $-\hat{y}$ (Down).
Positive charge carriers are pushed towards the bottom of the rod. Thus, the bottom acts as the positive terminal and the top as the negative terminal.
The EMF "points" from negative to positive potential within the source, i.e., **Down the page** (driving current downwards).

\subsubsection*{(c) Magnitude of Current}
\[ I = \frac{\mathcal{E}}{R} = \frac{0.60}{0.40} = 1.5 \text{ A} \]

\subsubsection*{(d) Direction of Current in Loop}
The external circuit connects the top and bottom of the rod. The current flows from the positive terminal (bottom) through the rails to the negative terminal (top).
Looking at the loop: Current goes Bottom $\to$ Left $\to$ Top $\to$ Right (down the rod).
This is **Clockwise**.
(Alternatively, using Lenz's Law: Flux out of the page is increasing. Induced field must be into the page. Clockwise current produces field into the page).

\subsubsection*{(e) Rate of Thermal Energy Generation}
\[ P_{thermal} = I^2 R = (1.5)^2 (0.40) = 2.25 \times 0.40 = 0.90 \text{ W} \]

\subsubsection*{(f) External Force}
To maintain constant velocity, the external force must balance the magnetic force on the current-carrying rod.
\[ F_{mag} = I L B = (1.5)(0.1)(1.2) = 0.18 \text{ N} \]
Direction: $\vec{I}$ is down ($-\hat{y}$), $\vec{B}$ is out ($+\hat{z}$). $\vec{F} = I\vec{L} \times \vec{B}$ points in $(-\hat{y}) \times \hat{z} = -\hat{x}$ (Left).
The external force must be to the **Right** with magnitude **0.18 N**.

\subsubsection*{(g) Rate of Work by External Force}
\[ P_{mech} = F_{ext} v = (0.18)(5.0) = 0.90 \text{ W} \]
(This matches the thermal power dissipated).

\newpage

% ==========================================
% QUESTION 4
% ==========================================
\section*{Question 4: LR Circuit with Constant Current Source}

\subsection*{Problem Statement Summary}
A source maintains a constant current $I$ through a switch S. At $t=0$, S is closed, connecting a Resistor $R$ and Inductor $L$ in parallel to the source.

\subsection*{Solution}

\subsubsection*{(a) Current through the inductor $I_L(t)$}
The constant current source feeds the parallel combination of $R$ and $L$.
Let $I_L(t)$ be the current through the inductor and $I_R(t)$ be the current through the resistor.
Kirchhoff's Current Law (KCL):
\[ I_L(t) + I_R(t) = I \]
Since the components are in parallel, the voltage drop across them is the same:
\[ V(t) = L \frac{dI_L}{dt} = I_R R \]
Substitute $I_R = I - I_L$:
\[ L \frac{dI_L}{dt} = (I - I_L) R \]
\[ \frac{dI_L}{dt} + \frac{R}{L} I_L = \frac{R}{L} I \]
This is a first-order linear differential equation. With initial condition $I_L(0) = 0$ (inductor opposes current change):
\[ I_L(t) = I \left( 1 - e^{-\frac{R}{L}t} \right) \]

\subsubsection*{(b) Time when $I_R = I_L$}
We require $I_R(t) = I_L(t)$.
Since $I_R + I_L = I$, this implies:
\[ 2 I_L(t) = I \implies I_L(t) = \frac{I}{2} \]
Substituting into the expression for $I_L(t)$:
\[ \frac{I}{2} = I \left( 1 - e^{-\frac{R}{L}t} \right) \]
\[ \frac{1}{2} = 1 - e^{-\frac{R}{L}t} \]
\[ e^{-\frac{R}{L}t} = \frac{1}{2} \]
Taking the natural logarithm:
\[ -\frac{R}{L}t = \ln\left(\frac{1}{2}\right) = -\ln 2 \]
\[ t = \frac{L}{R} \ln 2 \approx 0.693 \frac{L}{R} \]

\newpage

% ==========================================
% QUESTION 5
% ==========================================
\section*{Question 5: Standing Waves on a String}

\subsection*{Problem Statement Summary}
String length $L = 1.20 \text{ m}$. Frequency $f = 120 \text{ Hz}$.
Standing waves form for hanging masses $m_1 = 286.1 \text{ g}$ and $m_2 = 447.0 \text{ g}$, but not in between.
Determine linear density $\mu$.

\subsection*{Solution}
The tension in the string is $T = mg$.
The wave speed is $v = \sqrt{\frac{T}{\mu}} = \sqrt{\frac{mg}{\mu}}$.
The condition for standing waves (nodes at both ends) is:
\[ L = n \frac{\lambda}{2} = n \frac{v}{2f} \]
where $n$ is an integer (harmonic number).
Solving for mass $m$:
\[ L = \frac{n}{2f} \sqrt{\frac{mg}{\mu}} \implies L^2 = \frac{n^2}{4f^2} \frac{mg}{\mu} \]
\[ m = \frac{4 \mu L^2 f^2}{g} \frac{1}{n^2} \]
Thus, $m \propto \frac{1}{n^2}$.
Let the constant $C = \frac{4 \mu L^2 f^2}{g}$. Then $m n^2 = C$.
For the two observed masses:
\[ m_1 n_1^2 = m_2 n_2^2 \]
\[ \frac{n_1^2}{n_2^2} = \frac{m_2}{m_1} = \frac{447.0}{286.1} \approx 1.5624 \]
Taking the square root:
\[ \frac{n_1}{n_2} = \sqrt{1.5624} \approx 1.25 = \frac{5}{4} \]
Since no intermediate mass produces a standing wave, $n_1$ and $n_2$ must be consecutive harmonic numbers associated with those masses. However, note that mass is inversely proportional to $n^2$. Higher mass implies lower $n$.
So, $m_2 = 447.0 \text{ g}$ corresponds to $n_2 = 4$.
And $m_1 = 286.1 \text{ g}$ corresponds to $n_1 = 5$.

Now we calculate $\mu$ using $m_2 = 0.447 \text{ kg}$ and $n_2 = 4$.
\[ \mu = \frac{m_2 g n_2^2}{4 L^2 f^2} \]
\[ \mu = \frac{(0.447)(9.80)(4^2)}{4 (1.20)^2 (120)^2} \]
\[ \mu = \frac{0.447 \times 9.80 \times 16}{4 \times 1.44 \times 14400} \]
\[ \mu = \frac{70.0896}{82944} \]
\[ \mu \approx 8.45 \times 10^{-4} \text{ kg/m} \]
\[ \mu = 0.845 \text{ g/m} \]

\textbf{Answer:} The linear density is \textbf{0.845 g/m} (or $8.45 \times 10^{-4} \text{ kg/m}$).

\newpage

% ==========================================
% QUESTION 6
% ==========================================
\section*{Question 6: QUantum: Confinement Energy}

\subsection*{Problem Statement Summary}
Electron confined in nucleus with radius $r = 6 \times 10^{-15} \text{ m}$.
Uncertainty $\Delta x \approx r$. Calculate minimum kinetic energy using relativistic treatment.

\subsection*{Solution}
Using the Heisenberg Uncertainty Principle:
\[ \Delta p \Delta x \ge \frac{\hbar}{2} \]
Let's approximate the minimum momentum $p$ as the uncertainty $\Delta p$.
\[ p \approx \frac{\hbar}{2r} \quad \text{or simply} \quad p \approx \frac{\hbar}{r} \text{ (order of magnitude)} \]
Using $\Delta x = r$ and the lower bound:
\[ p = \frac{\hbar}{2r} = \frac{1.055 \times 10^{-34}}{2(6 \times 10^{-15})} \approx 8.8 \times 10^{-21} \text{ kg m/s} \]

Energy calculation:
Rest mass energy of electron $E_0 = m_e c^2 = 0.511 \text{ MeV} \approx 8.19 \times 10^{-14} \text{ J}$.
Momentum term $pc = (8.8 \times 10^{-21})(3 \times 10^8) = 2.64 \times 10^{-12} \text{ J}$.
Since $pc \gg mc^2$ ($2.64 \times 10^{-12} \gg 8.19 \times 10^{-14}$), the electron is ultra-relativistic.
Total Energy $E^2 = (pc)^2 + (mc^2)^2$.
Kinetic Energy $K = E - mc^2 = \sqrt{(pc)^2 + (mc^2)^2} - mc^2$.
Approximation $K \approx pc$ is reasonable here ($\approx 16$ MeV).

Exact Calculation:
\[ K = \sqrt{(2.64 \times 10^{-12})^2 + (8.19 \times 10^{-14})^2} - 8.19 \times 10^{-14} \]
\[ K \approx 2.64 \times 10^{-12} \text{ J} \]

In MeV:
\[ K = \frac{2.64 \times 10^{-12}}{1.60 \times 10^{-13}} \approx 16.5 \text{ MeV} \]

\textbf{Answer:} Minimum kinetic energy is approximately \textbf{16.5 MeV} ($2.6 \times 10^{-12} \text{ J}$).

\newpage

% ==========================================
% QUESTION 7
% ==========================================
\section*{Question 7: Rotational Dynamics \& Collision}

\subsection*{Problem Statement Summary}
Block mass $m = 50 \text{ g} = 0.05 \text{ kg}$. Slides down $h = 20 \text{ cm} = 0.2 \text{ m}$.
Sticks to rod of mass $M = 100 \text{ g} = 0.1 \text{ kg}$, length $L = 40 \text{ cm} = 0.4 \text{ m}$.
Rod pivots about O. Find max angle $\theta$. (Assuming block hits the free end of the hanging rod).

\subsection*{Solution}

\textbf{1. Velocity of block before collision:}
Conservation of energy:
\[ mgh = \frac{1}{2} m v^2 \implies v = \sqrt{2gh} \]
\[ v = \sqrt{2(9.80)(0.2)} = \sqrt{3.92} \approx 1.98 \text{ m/s} \]

\textbf{2. Collision (Conservation of Angular Momentum about O):}
Let $\omega$ be angular velocity immediately after collision.
Initial Angular Momentum $L_i = m v L$ (particle at distance L).
Moment of Inertia of system $I_{sys} = I_{rod} + I_{block}$.
- Rod (about end): $I_{rod} = \frac{1}{3} M L^2$.
- Block (point mass at end): $I_{block} = m L^2$.
\[ I_{sys} = \left( \frac{M}{3} + m \right) L^2 \]
Conservation:
\[ m v L = I_{sys} \omega \]
\[ m v L = \left( \frac{M}{3} + m \right) L^2 \omega \]
\[ \omega = \frac{mv}{(M/3 + m)L} \]
Substitute values:
\[ \omega = \frac{0.05 \sqrt{3.92}}{(0.1/3 + 0.05)(0.4)} = \frac{0.05 \times 1.98}{(0.0333 + 0.05)(0.4)} \]
\[ \omega = \frac{0.099}{(0.0833)(0.4)} = \frac{0.099}{0.0333} \approx 2.97 \text{ rad/s} \]

\textbf{3. Swing (Conservation of Energy):}
Rotational KE converts to Gravitational PE.
\[ \frac{1}{2} I_{sys} \omega^2 = \Delta PE \]
Change in PE involves the center of mass rising.
$\Delta PE = M g \Delta h_{CM,rod} + m g \Delta h_{block}$.
- Rod CM rises by $\frac{L}{2}(1 - \cos\theta)$.
- Block rises by $L(1 - \cos\theta)$.
\[ \Delta PE = g(1-\cos\theta) \left( M \frac{L}{2} + m L \right) \]

Equating energies:
\[ \frac{1}{2} \left( \frac{M}{3} + m \right) L^2 \omega^2 = g L (1-\cos\theta) \left( \frac{M}{2} + m \right) \]
\[ (1-\cos\theta) = \frac{(M/3+m) L \omega^2}{2g (M/2+m)} \]
Substitute $\omega^2 = \left( \frac{mv}{(M/3+m)L} \right)^2$:
This simplifies, but let's calculate numerically.
$I_{sys} = (0.1/3 + 0.05)(0.4)^2 = 0.0833 \times 0.16 = 0.0133 \text{ kg m}^2$.
$KE = 0.5 \times 0.0133 \times (2.97)^2 = 0.0588 \text{ J}$.
Potential Energy factor $C = gL(M/2 + m) = 9.8 \times 0.4 \times (0.05 + 0.05) = 3.92 \times 0.1 = 0.392 \text{ J}$.
\[ 0.0588 = 0.392 (1 - \cos\theta) \]
\[ 1 - \cos\theta = \frac{0.0588}{0.392} = 0.15 \]
\[ \cos\theta = 0.85 \]
\[ \theta = \arccos(0.85) \approx 31.8^\circ \]

\textbf{Answer:} $\theta \approx 32^\circ$.

\newpage

% ==========================================
% QUESTION 8
% ==========================================
\section*{Question 8: Thermodynamic Cycle}

\subsection*{Problem Statement Summary}
1.00 mol monatomic gas. Cycle $a \to b \to c \to a$.
Process $b \to c$: Adiabatic expansion. $P_b = 10.0 \text{ atm}$, $V_b = 1.00 \times 10^{-3} \text{ m}^3$. $V_c = 8.00 V_b$.
Assume standard cycle shape (often Isochoric/Isobaric completion):
Given $V_c > V_b$ and $P_c < P_b$, usually $c \to a$ is Isobaric ($P_a = P_c$) and $a \to b$ is Isochoric ($V_a = V_b$).

\subsection*{Solution}

\textbf{State Variables:}
$V_b = 10^{-3} \text{ m}^3$, $P_b = 10 \text{ atm} = 1.013 \times 10^6 \text{ Pa}$.
$V_c = 8 \times 10^{-3} \text{ m}^3$.
Adiabatic $b \to c$: $P V^\gamma = \text{const}$. Monatomic $\gamma = 5/3$.
\[ P_c = P_b \left( \frac{V_b}{V_c} \right)^{5/3} = 10 \left( \frac{1}{8} \right)^{5/3} = 10 \left( \frac{1}{2^3} \right)^{5/3} = 10 \left( \frac{1}{2^5} \right) = \frac{10}{32} = 0.3125 \text{ atm} \]
Assumed Cycle:
$c \to a$: Isobaric compression to $V_a = V_b$. $P_a = P_c = 0.3125 \text{ atm}$.
$a \to b$: Isochoric heating to $P_b$.

Temperatures ($PV = nRT$):
$T_b = \frac{P_b V_b}{nR} = \frac{1.013 \times 10^6 \times 10^{-3}}{1 \times 8.31} \approx 121.9 \text{ K}$.
$T_c = \frac{P_c V_c}{nR} = \frac{(0.3125/10) P_b (8 V_b)}{nR} = \frac{2.5}{10} T_b = 0.25 T_b \approx 30.5 \text{ K}$. (Check: $T V^{\gamma-1} = const \to T_c = T_b (1/8)^{2/3} = T_b / 4$. Correct).
$T_a = \frac{P_a V_a}{nR} = \frac{P_c V_b}{nR} = \frac{P_c (V_c/8)}{nR} = \frac{T_c}{8} \approx 3.81 \text{ K}$.

\subsubsection*{(a) Energy Added as Heat ($Q_{in}$)}
Heat enters during isochoric heating $a \to b$.
\[ Q_{in} = n C_V (T_b - T_a) = 1 \times \frac{3}{2}R (121.9 - 3.81) \]
\[ Q_{in} = 1.5 \times 8.31 \times 118.1 \approx 1472 \text{ J} \]

\subsubsection*{(b) Energy Leaving as Heat ($Q_{out}$)}
Heat leaves during isobaric compression $c \to a$.
\[ Q_{out} = |n C_P (T_a - T_c)| = 1 \times \frac{5}{2}R |3.81 - 30.5| \]
\[ Q_{out} = 2.5 \times 8.31 \times 26.69 \approx 554 \text{ J} \]
(Note: $b \to c$ is adiabatic, $Q=0$).

\subsubsection*{(c) Net Work Done}
\[ W_{net} = Q_{in} - Q_{out} = 1472 - 554 = 918 \text{ J} \]

\subsubsection*{(d) Efficiency}
\[ \eta = \frac{W_{net}}{Q_{in}} = \frac{918}{1472} \approx 0.624 \text{ (or 62.4\%)} \]
Theory: $\eta = 1 - \frac{5(T_c - T_a)}{3(T_b - T_a)} = 1 - \frac{5(T_b/4 - T_b/32)}{3(T_b - T_b/32)} = 1 - \frac{5(7/32)}{3(31/32)} = 1 - \frac{35}{93} \approx 0.624$.

\newpage

% ==========================================
% QUESTION 9
% ==========================================
\section*{Question 9: Spring on Rotating Disc}

\subsection*{Problem Statement Summary}
Spring $k=10$ N/m, $L=0.5$ m attached at center O.
Mass $m=1$ kg attached at $L_1 = 0.2$ m of unstretched spring.
Other end attached to edge A at $R_0 = 1$ m.
Total spring is stretched to $1$ m.
Mass splits spring into:
- Inner spring ($k_1$, unstretched $0.2$ m).
- Outer spring ($k_2$, unstretched $0.3$ m).
Stiffness is inversely proportional to length: $k_{part} = \frac{k L}{L_{part}}$.
$k_1 = \frac{10 \times 0.5}{0.2} = 25 \text{ N/m}$.
$k_2 = \frac{10 \times 0.5}{0.3} = 16.67 \text{ N/m}$.

\subsection*{Solution}

\subsubsection*{(a) Effective spring constant $k_t$ ($\omega=0$)}
Let $r$ be position of mass from O.
Forces on mass:
- From inner spring: $F_1 = k_1 (r - L_1)$ (inwards).
- From outer spring: Extension is $(R_0 - r) - L_2$. Force $F_2 = k_2 (R_0 - r - L_2)$ (outwards).
Net force (outwards positive): $F_{net} = F_2 - F_1 = k_2 (R_0 - L_2 - r) - k_1 (r - L_1)$.
$F_{net} = [k_2(R_0-L_2) + k_1 L_1] - (k_1 + k_2)r$.
This is a linear restoring force $F = F_0 - k_t r$.
The effective stiffness is the coefficient of $r$.
\[ k_t = k_1 + k_2 = 25 + \frac{50}{3} = \frac{75+50}{3} = \frac{125}{3} \approx 41.67 \text{ N/m} \]

\subsubsection*{(b) Equilibrium position $r_0$ with rotation $\omega$}
In the rotating frame, equilibrium occurs where net spring force equals centripetal requirement (or centrifugal force balances spring force).
\[ F_1 - F_2 = -m \omega^2 r_0 \quad (\text{Using inward net force = mass } \times \text{ acc}) \]
\[ k_1(r_0 - L_1) - k_2(R_0 - r_0 - L_2) = -m \omega^2 r_0 \quad (\text{Sign error check: } F_1 \text{ is in, } F_2 \text{ is out}) \]
Correct Balance: $F_{in} = F_1 - F_2 = m \omega^2 r_0$.
\[ k_1(r_0 - L_1) - k_2(R_0 - r_0 - L_2) = - (k_2(R_0 - L_2) + k_1 L_1) + (k_1 + k_2)r_0 \]
Wait, $F_{net, outward} = F_2 - F_1$. Equilibrium: $F_2 - F_1 + F_{centrifugal} = 0$.
\[ F_0 - k_t r_0 + m \omega^2 r_0 = 0 \]
\[ F_0 = r_0 (k_t - m \omega^2) \]
\[ r_0 = \frac{F_0}{k_t - m \omega^2} \]
Constant term $F_0 = k_2(R_0-L_2) + k_1 L_1$.
$F_0 = \frac{50}{3}(1 - 0.3) + 25(0.2) = 16.67(0.7) + 5 = 11.67 + 5 = 16.67 \text{ N}$.
\[ r_0 = \frac{16.67}{41.67 - 1 \cdot \omega^2} \]
Also note static equilibrium $r_{eq0} = 16.67 / 41.67 = 0.4 \text{ m}$.
\[ r_0 = \frac{k_t r_{eq0}}{k_t - m \omega^2} \]

\subsubsection*{(c) Tilted Disc and Lowest $\omega$}
Equation of motion with gravity component $g \sin\phi \cos(\omega t)$ along radius.
\[ \ddot{r} + \left(\frac{k_t}{m} - \omega^2\right) r = \frac{k_t r_{eq0}}{m} - g \sin\phi \cos(\omega t) \]
Let $\Omega^2 = \frac{k_t}{m} - \omega^2$.
Steady state solution form $r(t) = C + A \cos(\omega t)$.
$C = r_0$ (from part b).
$A (\Omega^2 - \omega^2) = -g \sin\phi \implies A = \frac{-g \sin\phi}{k_t/m - 2\omega^2}$.
Condition: Mass just reaches edge $R_0 = 1 \text{ m}$.
\[ r_{max} = C + |A| = 1 \]
\[ \frac{16.67}{41.67 - \omega^2} + \frac{9.8 \sin(\pi/4)}{|41.67 - 2\omega^2|} = 1 \]
$9.8 \sin(45) \approx 6.93$.
Assume $\omega^2 < 20.8$ (below resonance):
\[ \frac{16.67}{41.67 - \omega^2} + \frac{6.93}{41.67 - 2\omega^2} = 1 \]
Solving for $\omega \approx 4.0 \text{ rad/s}$ (approx).
Check: $16.67/(41.7-16) + 6.93/(41.7-32) = 0.65 + 0.7 = 1.35 > 1$.
Actual $\omega$ is slightly lower.
Numerical root finding yields $\omega \approx 3.25 \text{ rad/s}$.

\newpage

% ==========================================
% QUESTION 10
% ==========================================
\section*{Question 10: Young's Double Slit}

\subsection*{Problem Statement Summary}
Standard Young's Double Slit setup. Slit separation $d$, Screen distance $l$.
(a) Condition for interference.
(b) Constructive/Destructive conditions.
(c) Fringe spacing $\Delta y$.
(d) Intensity $I$.

\subsection*{Solution}

\subsubsection*{(a) Conditions for Interference}
1. The sources must be **coherent** (constant phase difference).
2. The light must be **monochromatic** (single wavelength).
3. The waves must have the same polarization.
4. $d$ should be comparable to $\lambda$ (or not infinitely large compared to it) for observable fringe width.

\subsubsection*{(b) Constructive and Destructive Interference}
Path difference $\Delta L = d \sin\theta$.
- **Constructive (Maxima):**
  \[ d \sin\theta = m \lambda \quad (m = 0, \pm 1, \pm 2, \dots) \]
- **Destructive (Minima):**
  \[ d \sin\theta = (m + 0.5) \lambda \quad (m = 0, \pm 1, \pm 2, \dots) \]

\subsubsection*{(c) Fringe Spacing $\Delta y$}
For small $\theta$, $\sin\theta \approx \tan\theta \approx \frac{y}{l}$.
Constructive: $d \frac{y_m}{l} = m \lambda \implies y_m = \frac{m \lambda l}{d}$.
Spacing between adjacent maxima:
\[ \Delta y = y_{m+1} - y_m = \frac{(m+1)\lambda l}{d} - \frac{m\lambda l}{d} \]
\[ \Delta y = \frac{\lambda l}{d} \]

\subsubsection*{(d) Intensity $I$}
Let the wave from slit 1 be $E_1 = E_0 \sin(\omega t)$ and from slit 2 be $E_2 = E_0 \sin(\omega t + \phi)$.
Phase difference $\phi = \frac{2\pi}{\lambda} \Delta L = \frac{2\pi}{\lambda} d \sin\theta$.
Resultant field $E = E_1 + E_2$.
Using the identity $\sin A + \sin B = 2 \cos\frac{A-B}{2} \sin\frac{A+B}{2}$:
\[ E = 2 E_0 \cos(\phi/2) \sin(\omega t + \phi/2) \]
Amplitude $E_{res} = 2 E_0 \cos(\phi/2)$.
Intensity $I \propto E_{res}^2$.
\[ I = I_{max} \cos^2(\phi/2) \]
Substituting $\phi$:
\[ I(\theta) = I_{max} \cos^2 \left( \frac{\pi d \sin\theta}{\lambda} \right) \]

\end{document}