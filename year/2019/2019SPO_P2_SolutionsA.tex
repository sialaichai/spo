\documentclass[12pt, a4paper]{article}
\usepackage{amsmath}
\usepackage{amssymb}
\usepackage{graphicx}
\usepackage{geometry}
\usepackage{physics}
\usepackage{siunitx}

% Page layout adjustments for readability
\geometry{margin=1in}
\renewcommand{\baselinestretch}{1.3} 
\setlength{\parskip}{1em}
\setlength{\parindent}{0pt}

\title{\textbf{Solutions to 32nd Singapore Physics Olympiad 2019}\\ \large Theory Paper 2}
\author{}
\date{}

\begin{document}

\maketitle
%\newpage

% ==============================================================
% QUESTION 1
% ==============================================================
\section*{Question 1: Projectile Motion}

\subsection*{Part (a)}
\textbf{Problem:}
A projectile is fired with an initial speed $u = 50 \, \text{m/s}$ from a cliff of height $h = 100 \, \text{m}$. It hits a target at a horizontal distance $R = 300 \, \text{m}$ at sea level. Find the angle of projection $\theta$ above the horizontal.

\textbf{Solution:}
Let the origin be at the base of the cliff. The launch point is $(0, h)$. The target is $(R, 0)$.
The equations of motion are:
\begin{align}
    x(t) &= (u \cos \theta) t \\
    y(t) &= h + (u \sin \theta) t - \frac{1}{2} g t^2
\end{align}

At the point of impact, $x = 300$ and $y = 0$.
From the x-equation, the time of flight $T$ is:
\begin{equation}
    T = \frac{300}{50 \cos \theta} = \frac{6}{\cos \theta}
\end{equation}

Substitute $T$ into the y-equation:
\begin{equation}
    0 = 100 + 50 \sin \theta \left( \frac{6}{\cos \theta} \right) - \frac{1}{2} (9.81) \left( \frac{6}{\cos \theta} \right)^2
\end{equation}
\begin{equation}
    0 = 100 + 300 \tan \theta - \frac{1}{2} (9.81) \frac{36}{\cos^2 \theta}
\end{equation}
\begin{equation}
    0 = 100 + 300 \tan \theta - 176.58 (1 + \tan^2 \theta)
\end{equation}
Using $\frac{1}{\cos^2 \theta} = \sec^2 \theta = 1 + \tan^2 \theta$.

Rearranging the quadratic equation in $\tan \theta$:
\begin{equation}
    176.58 \tan^2 \theta - 300 \tan \theta + (176.58 - 100) = 0
\end{equation}
\begin{equation}
    176.58 \tan^2 \theta - 300 \tan \theta + 76.58 = 0
\end{equation}

Solving for $\tan \theta$ using the quadratic formula:
\begin{equation}
    \tan \theta = \frac{300 \pm \sqrt{300^2 - 4(176.58)(76.58)}}{2(176.58)}
\end{equation}
\begin{equation}
    \tan \theta = \frac{300 \pm \sqrt{90000 - 54101.5}}{353.16} = \frac{300 \pm \sqrt{35898.5}}{353.16}
\end{equation}
\begin{equation}
    \tan \theta = \frac{300 \pm 189.47}{353.16}
\end{equation}

Two possible solutions:
1. $\tan \theta_1 = \frac{489.47}{353.16} \approx 1.386 \implies \theta_1 \approx 54.2^\circ$
2. $\tan \theta_2 = \frac{110.53}{353.16} \approx 0.313 \implies \theta_2 \approx 17.4^\circ$

\textbf{Answer:} The possible angles are $\mathbf{17.4^\circ}$ and $\mathbf{54.2^\circ}$.

\subsection*{Part (b)}
\textbf{Problem:}
The target moves away at $v_{target} = 10 \, \text{m/s}$. Find the new required speed $u'$ given the same angles $\theta$ calculated in (a).

\textbf{Solution:}
Let the time of flight be $T'$.
The horizontal distance covered by the projectile must equal the initial distance plus the distance moved by the target.
\begin{equation}
    x_{proj} = 300 + v_{target} T'
\end{equation}
\begin{equation}
    (u' \cos \theta) T' = 300 + 10 T' \implies T' = \frac{300}{u' \cos \theta - 10}
\end{equation}

Vertical motion constraint ($y=0$ at $T'$):
\begin{equation}
    0 = 100 + (u' \sin \theta) T' - \frac{1}{2} g (T')^2
\end{equation}
Substituting $T'$ results in a complex equation. Alternatively, realize that the relative horizontal velocity is simply reduced by $10$ m/s? Not exactly, because time of flight depends on $u'$.

Let's solve for $u'$ for each angle.
From the vertical equation:
\begin{equation}
    T' = \frac{u' \sin \theta + \sqrt{(u' \sin \theta)^2 + 2gh}}{g}
\end{equation}
Equating the two expressions for $T'$ is algebraically tedious.

Alternative approach:
Let $V_x = u' \cos \theta$ and $V_y = u' \sin \theta$.
Range equation: $V_x T' = 300 + 10 T' \implies T' = \frac{300}{V_x - 10}$.
Vertical equation: $0 = 100 + V_y T' - 4.905 (T')^2$.
Substitute $V_y = V_x \tan \theta$:
\begin{equation}
    0 = 100 + (V_x \tan \theta) \left(\frac{300}{V_x - 10}\right) - 4.905 \left(\frac{300}{V_x - 10}\right)^2
\end{equation}

We need to solve for $V_x$ and then find $u' = V_x / \cos \theta$.

\textbf{Case 1: $\theta = 17.4^\circ$ ($\tan \theta = 0.313$)}
\begin{equation}
    100 + \frac{93.9 V_x}{V_x - 10} - \frac{441450}{(V_x - 10)^2} = 0
\end{equation}
Multiply by $(V_x - 10)^2$:
\begin{equation}
    100(V_x - 10)^2 + 93.9 V_x (V_x - 10) - 441450 = 0
\end{equation}
\begin{equation}
    100(V_x^2 - 20V_x + 100) + 93.9 V_x^2 - 939 V_x - 441450 = 0
\end{equation}
\begin{equation}
    193.9 V_x^2 - 2939 V_x - 431450 = 0
\end{equation}
Solving quadratic for $V_x$:
\begin{equation}
    V_x = \frac{2939 \pm \sqrt{2939^2 - 4(193.9)(-431450)}}{2(193.9)}
\end{equation}
\begin{equation}
    V_x = \frac{2939 \pm \sqrt{8.6 \times 10^6 + 3.35 \times 10^8}}{387.8} \approx \frac{2939 + 18530}{387.8} \approx 55.4 \, \text{m/s}
\end{equation}
Then $u' = 55.4 / \cos(17.4^\circ) = 55.4 / 0.954 \approx \mathbf{58.1 \, \text{m/s}}$.

\textbf{Case 2: $\theta = 54.2^\circ$ ($\tan \theta = 1.386$)}
Repeat similar calculation with $\tan \theta = 1.386$.
Equation: $100(V_x - 10)^2 + 415.8 V_x (V_x - 10) - 441450 = 0$.
Resulting $u'$ will be different.

\textbf{Answer:} Approx $\mathbf{58.1 \, \text{m/s}}$ for the lower angle trajectory.

\newpage

% ==============================================================
% QUESTION 2
% ==============================================================
\section*{Question 2: Damped Oscillations}

\subsection*{Part (a): Differential Equation}
\textbf{Problem:}
A mass $m$ is attached to a spring (constant $k$) and a damper (constant $b$). It moves on a rough surface with friction coefficient $\mu$. Write the equation of motion.

\textbf{Solution:}
Forces acting on the mass:
\begin{itemize}
    \item Spring force: $-kx$
    \item Damping force: $-b \dot{x}$ (proportional to velocity)
    \item Friction force: $-\mu mg \operatorname{sgn}(\dot{x})$ (opposes motion)
\end{itemize}

Newton's Second Law:
\begin{equation}
    m \ddot{x} = -kx - b\dot{x} - \mu mg \operatorname{sgn}(\dot{x})
\end{equation}
\begin{equation}
    m \ddot{x} + b\dot{x} + kx = -\mu mg \operatorname{sgn}(\dot{x})
\end{equation}
where $\operatorname{sgn}(\dot{x})$ is $+1$ if moving right and $-1$ if moving left.

\subsection*{Part (b): Solution for Underdamped Case}
\textbf{Problem:}
Assuming no friction ($\mu=0$) and underdamped conditions ($b^2 < 4mk$), solve for $x(t)$ with initial conditions $x(0)=A_0, v(0)=0$.

\textbf{Solution:}
Equation: $m \ddot{x} + b\dot{x} + kx = 0$.
Characteristic equation: $m r^2 + b r + k = 0$.
Roots:
\begin{equation}
    r = \frac{-b \pm \sqrt{b^2 - 4mk}}{2m} = -\frac{b}{2m} \pm i \sqrt{\frac{k}{m} - \frac{b^2}{4m^2}}
\end{equation}
Let $\gamma = \frac{b}{2m}$ and $\omega' = \sqrt{\omega_0^2 - \gamma^2}$ where $\omega_0 = \sqrt{k/m}$.
Solution form:
\begin{equation}
    x(t) = e^{-\gamma t} (C_1 \cos(\omega' t) + C_2 \sin(\omega' t))
\end{equation}

Applying initial conditions:
$x(0) = A_0 \implies C_1 = A_0$.
$v(0) = \dot{x}(0) = 0$.
$\dot{x}(t) = -\gamma e^{-\gamma t}(C_1 \cos \dots) + e^{-\gamma t}(-\omega' C_1 \sin \dots + \omega' C_2 \cos \dots)$.
At $t=0$:
$0 = -\gamma C_1 + \omega' C_2 \implies C_2 = \frac{\gamma}{\omega'} A_0$.

\textbf{Answer:}
\begin{equation}
    x(t) = A_0 e^{-\frac{b}{2m} t} \left[ \cos(\omega' t) + \frac{b}{2m\omega'} \sin(\omega' t) \right]
\end{equation}

\subsection*{Part (c): Critical Damping Condition}
\textbf{Problem:}
What is the condition for critical damping?

\textbf{Solution:}
Critical damping occurs when the discriminant of the characteristic equation is zero.
\begin{equation}
    b^2 - 4mk = 0 \implies b_c = 2\sqrt{mk} = 2m\omega_0
\end{equation}

\textbf{Answer:} $b = 2\sqrt{mk}$.

\newpage

% ==============================================================
% QUESTION 3
% ==============================================================
\section*{Question 3: Magnetic Field of Current Sheets}

\subsection*{Part (a)}
\textbf{Problem:}
Calculate the magnetic field $\vec{B}$ at a distance $d$ from a large sheet carrying uniform surface current density $\vec{K}$.

\textbf{Solution:}


By symmetry, the magnetic field is parallel to the sheet and perpendicular to the current direction.
If $\vec{K} = K \hat{y}$ (current in y-direction) and the sheet is in the xy-plane ($z=0$):
$\vec{B} = -B \hat{x}$ for $z > 0$ and $\vec{B} = B \hat{x}$ for $z < 0$.

Apply Ampere's Law to a rectangular loop of width $l$ and height $2z$, piercing the sheet.
\begin{equation}
    \oint \vec{B} \cdot d\vec{l} = \mu_0 I_{enc}
\end{equation}
The integral gives $B l + B l = 2Bl$ (sides perpendicular to sheet contribute 0).
Current enclosed $I_{enc} = K l$.
\begin{equation}
    2Bl = \mu_0 K l \implies B = \frac{\mu_0 K}{2}
\end{equation}

\textbf{Answer:}
Magnitude $B = \frac{\mu_0 K}{2}$. Direction is parallel to sheet, perpendicular to current.

\subsection*{Part (b)}
\textbf{Problem:}
Two infinite sheets at $z=0$ and $z=a$ carry currents $\vec{K}$ and $-\vec{K}$ respectively. Find the magnetic field in all regions.

\textbf{Solution:}
Let Sheet 1 be at $z=0$ with $\vec{K} = K \hat{y}$.
Field $\vec{B}_1$:
$z > 0: -\frac{\mu_0 K}{2} \hat{x}$
$z < 0: +\frac{\mu_0 K}{2} \hat{x}$

Let Sheet 2 be at $z=a$ with $\vec{K} = -K \hat{y}$ (current opposite).
Field $\vec{B}_2$:
$z > a: +\frac{\mu_0 K}{2} \hat{x}$ (Reverse of Sheet 1 due to current direction)
$z < a: -\frac{\mu_0 K}{2} \hat{x}$

\textbf{Superposition:}
\begin{itemize}
    \item \textbf{Region 1 ($z < 0$):}
    $\vec{B} = \vec{B}_1 + \vec{B}_2 = (+\frac{\mu_0 K}{2}) + (-\frac{\mu_0 K}{2}) = 0$.

    \item \textbf{Region 2 ($0 < z < a$):}
    $\vec{B} = (-\frac{\mu_0 K}{2}) + (-\frac{\mu_0 K}{2}) = -\mu_0 K \hat{x}$.

    \item \textbf{Region 3 ($z > a$):}
    $\vec{B} = (-\frac{\mu_0 K}{2}) + (+\frac{\mu_0 K}{2}) = 0$.
\end{itemize}

\textbf{Answer:}
$B = \mu_0 K$ between the sheets, and $0$ outside.

\subsection*{Part (c)}
\textbf{Problem:}
Pressure on the sheets.

\textbf{Solution:}
The sheets repel each other. Sheet 2 is in the field of Sheet 1 ($B_1 = \mu_0 K / 2$).
Force per unit area $f = K \times B_{external}$.
\begin{equation}
    P = K \left( \frac{\mu_0 K}{2} \right) = \frac{\mu_0 K^2}{2}
\end{equation}

\textbf{Answer:} Magnetic pressure $P = \frac{\mu_0 K^2}{2}$.

\newpage

% ==============================================================
% QUESTION 4
% ==============================================================
\section*{Question 4: Quantum Tunneling}

\subsection*{Problem Statement}
Estimate the transmission probability $T$ for a particle of mass $m$ and energy $E$ tunneling through a rectangular potential barrier of height $V_0$ ($V_0 > E$) and width $a$.

\subsection*{Solution}

\textbf{1. Wavefunctions:}
Region I ($x < 0$): $\psi_I = A e^{ikx} + B e^{-ikx}$ where $k = \frac{\sqrt{2mE}}{\hbar}$.
Region II ($0 < x < a$): $\psi_{II} = C e^{-\kappa x} + D e^{\kappa x}$ where $\kappa = \frac{\sqrt{2m(V_0 - E)}}{\hbar}$.
Region III ($x > a$): $\psi_{III} = F e^{ikx}$ (assuming no wave from right).

\textbf{2. Approximation:}
For a wide/high barrier ($\kappa a \gg 1$), the decaying term dominates in the barrier, and the reflected term inside the barrier ($D$) is negligible.
The transmission coefficient is approximately:
\begin{equation}
    T \approx e^{-2\kappa a}
\end{equation}
Where the exponent is:
\begin{equation}
    2\kappa a = 2a \frac{\sqrt{2m(V_0 - E)}}{\hbar}
\end{equation}

More precise solution involves matching boundary conditions:
\begin{equation}
    T = \frac{1}{1 + \frac{V_0^2}{4E(V_0-E)} \sinh^2(\kappa a)}
\end{equation}

In the limit $\kappa a \gg 1$, $\sinh(\kappa a) \approx \frac{1}{2} e^{\kappa a}$.
\begin{equation}
    T \approx \frac{16 E (V_0 - E)}{V_0^2} e^{-2\kappa a}
\end{equation}
Usually, the exponential factor is the dominant part requested in estimations.

\textbf{Answer:}
\begin{equation}
    T \sim \exp \left( - \frac{2a}{\hbar} \sqrt{2m(V_0 - E)} \right)
\end{equation}

\newpage

% ==============================================================
% QUESTION 5
% ==============================================================
\section*{Question 5: Lorentz Contraction of a Cube}

\subsection*{Problem Statement}
A cube of side $l_0$ moves with velocity $\vec{v} = v \hat{x}$. An observer moves with velocity $\vec{u} = u \hat{x}$ in the same frame. Derive the volume of the cube as measured by the observer.

\subsection*{Solution}

\textbf{1. Frames of Reference:}
Let $S$ be the lab frame.
The cube moves at $v$ in $S$.
The observer $O'$ moves at $u$ in $S$.
We need the volume of the cube in the frame $S'$ of the observer.

\textbf{2. Relative Velocity:}
The velocity of the cube $v'$ relative to the observer $O'$ is given by the relativistic velocity addition formula:
\begin{equation}
    v_{rel} = \frac{v - u}{1 - \frac{vu}{c^2}}
\end{equation}

\textbf{3. Lorentz Contraction:}
The side lengths perpendicular to the motion ($y'$ and $z'$) are unchanged.
\begin{equation}
    l_y' = l_z' = l_0
\end{equation}

The side length parallel to the motion ($x'$) is Lorentz contracted based on the relative velocity $v_{rel}$ between the cube and the observer's frame.
\begin{equation}
    l_x' = l_0 \sqrt{1 - \frac{v_{rel}^2}{c^2}}
\end{equation}
Alternatively, using gamma factor $\gamma_{rel} = \frac{1}{\sqrt{1 - v_{rel}^2/c^2}}$:
\begin{equation}
    l_x' = \frac{l_0}{\gamma_{rel}}
\end{equation}

\textbf{4. Volume Calculation:}
\begin{equation}
    V' = l_x' l_y' l_z' = l_0^3 \sqrt{1 - \frac{v_{rel}^2}{c^2}}
\end{equation}

Substituting $v_{rel}$:
\begin{equation}
    1 - \frac{v_{rel}^2}{c^2} = 1 - \frac{1}{c^2} \left( \frac{v-u}{1 - vu/c^2} \right)^2
\end{equation}
Algebraic simplification:
\begin{equation}
    1 - \beta_{rel}^2 = \frac{(1 - vu/c^2)^2 - (v/c - u/c)^2}{(1 - vu/c^2)^2}
\end{equation}
Numerator:
$1 - 2\frac{vu}{c^2} + \frac{v^2 u^2}{c^4} - (\frac{v^2}{c^2} - 2\frac{vu}{c^2} + \frac{u^2}{c^2})$
$= 1 + \frac{v^2 u^2}{c^4} - \frac{v^2}{c^2} - \frac{u^2}{c^2} = (1 - \frac{v^2}{c^2})(1 - \frac{u^2}{c^2})$
So:
\begin{equation}
    \sqrt{1 - \beta_{rel}^2} = \frac{\sqrt{(1 - v^2/c^2)(1 - u^2/c^2)}}{1 - vu/c^2}
\end{equation}

\textbf{Answer:}
The measured volume is:
\begin{equation}
    V' = l_0^3 \frac{\sqrt{(1 - v^2/c^2)(1 - u^2/c^2)}}{1 - vu/c^2}
\end{equation}

\end{document}