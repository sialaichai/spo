\documentclass[a4paper,12pt]{article}
\usepackage[utf8]{inputenc}
\usepackage[T1]{fontenc}
\usepackage{amsmath}
\usepackage{amssymb}
\usepackage{graphicx}
\usepackage{geometry}
\usepackage{siunitx}
\usepackage{fancyhdr}

% Page Setup
\geometry{margin=1in}
\pagestyle{fancy}
\fancyhf{}
\lhead{SPhO 2019 Theory Paper 2 Solutions}
\rhead{\thepage}

\title{\textbf{32\textsuperscript{nd} Singapore Physics Olympiad (SPhO) 2019} \\ \Large Theory Paper 2 Solutions}
\author{}
\date{}

\begin{document}

\maketitle
%\tableofcontents
%\newpage

% ==========================================
% QUESTION 1
% ==========================================
\section*{Question 1: Projectile Motion}

\subsection*{(a) Angle of Projection}
A projectile is fired with initial speed $v_0 = \SI{50}{m.s^{-1}}$ from a height $h = \SI{100}{m}$. The target is at horizontal distance $R = \SI{300}{m}$. We need to find the angle of inclination $\theta$.

Let the origin be at the bottom of the cliff. The launch point is $(0, h)$ and the target is at $(R, 0)$.
The equations of motion are:
\[ x(t) = (v_0 \cos\theta) t \]
\[ y(t) = h + (v_0 \sin\theta) t - \frac{1}{2} g t^2 \]

At the point of impact, $x = R$ and $y = 0$:
1. $t = \frac{R}{v_0 \cos\theta}$
2. $0 = h + v_0 \sin\theta \left( \frac{R}{v_0 \cos\theta} \right) - \frac{1}{2} g \left( \frac{R}{v_0 \cos\theta} \right)^2$

Substituting $R=300$, $h=100$, $v_0=50$, $g=9.81$:
\[ 0 = 100 + 300 \tan\theta - \frac{9.81 (300)^2}{2 (50)^2 \cos^2\theta} \]
\[ 0 = 100 + 300 \tan\theta - \frac{9.81 \times 90000}{5000} \sec^2\theta \]
\[ 0 = 100 + 300 \tan\theta - 176.58 (1 + \tan^2\theta) \]

Let $u = \tan\theta$. The equation becomes a quadratic in $u$:
\[ 176.58 u^2 - 300 u + (176.58 - 100) = 0 \]
\[ 176.58 u^2 - 300 u + 76.58 = 0 \]

Solving for $u$:
\[ u = \frac{300 \pm \sqrt{300^2 - 4(176.58)(76.58)}}{2(176.58)} \]
\[ u = \frac{300 \pm \sqrt{90000 - 54089}}{353.16} = \frac{300 \pm \sqrt{35911}}{353.16} \]
\[ u = \frac{300 \pm 189.5}{353.16} \]

Two solutions:
1. $u_1 = \frac{489.5}{353.16} \approx 1.386 \implies \theta_1 \approx 54.2^\circ$
2. $u_2 = \frac{110.5}{353.16} \approx 0.313 \implies \theta_2 \approx 17.4^\circ$

Usually, the lower angle (direct shot) is preferred unless specified, but both are valid mathematical solutions for hitting a stationary target.

\textbf{Answer:} $\theta \approx 17.4^\circ$ or $54.2^\circ$.

\subsection*{(b) Moving Target}
The target moves away at $v_T = \SI{10}{m.s^{-1}}$. The angle $\theta$ remains the same (assume $\theta \approx 17.4^\circ$, i.e., $\tan\theta = 0.313$). Find the new launch speed $v'$.

Let $t$ be the time of flight.
Target position: $x_T(t) = 300 + 10 t$.
Projectile position: $x_p(t) = (v' \cos\theta) t$.
Impact condition $x_p(t) = x_T(t)$:
\[ (v' \cos\theta) t = 300 + 10 t \implies t = \frac{300}{v' \cos\theta - 10} \]

Vertical motion constraint ($y=0$ at impact):
\[ 0 = 100 + (v' \sin\theta) t - 4.905 t^2 \]
Substitute $v' \sin\theta = (v' \cos\theta) \tan\theta$:
\[ 0 = 100 + (v' \cos\theta \cdot t) \tan\theta - 4.905 t^2 \]
Substitute $v' \cos\theta \cdot t = 300 + 10t$:
\[ 0 = 100 + (300 + 10t) \tan\theta - 4.905 t^2 \]

Using $\tan\theta = 0.313$:
\[ 0 = 100 + (300)(0.313) + 3.13 t - 4.905 t^2 \]
\[ 0 = 100 + 93.9 + 3.13 t - 4.905 t^2 \]
\[ 4.905 t^2 - 3.13 t - 193.9 = 0 \]

Solving for $t$:
\[ t = \frac{3.13 \pm \sqrt{3.13^2 - 4(4.905)(-193.9)}}{9.81} \]
\[ t = \frac{3.13 \pm \sqrt{9.8 + 3804}}{9.81} \approx \frac{3.13 + 61.76}{9.81} \approx 6.61 \text{ s} \]

Now find $v'$:
\[ v' \cos\theta - 10 = \frac{300}{6.61} = 45.39 \]
\[ v' \cos(17.4^\circ) = 55.39 \]
\[ v' = \frac{55.39}{0.954} \approx 58.0 \text{ m/s} \]

(If the higher angle $54.2^\circ$ was used, the calculation would yield a different speed).

\textbf{Answer:} Approximately \textbf{58.0 m/s}.

\newpage

% ==========================================
% QUESTION 2
% ==========================================
\section*{Question 2: Superposition}

\subsection*{(2a) Sound Interference}
Source $S$, Detector $D$, distance $L = \SI{120}{m}$. Wavelength $\lambda = \SI{1.33}{m}$.
Reflector at perpendicular distance $y$ from the line $SD$.
The path of the reflected ray (Ray 2) corresponds to the distance from the image source $S'$ to $D$.
$S'$ is at distance $2y$ from $S$ perpendicular to $SD$.
Path length of reflected ray: $L_2 = \sqrt{L^2 + (2y)^2}$.
Path length of direct ray: $L_1 = L = \SI{120}{m}$.
Path difference: $\Delta x = \sqrt{120^2 + 4y^2} - 120$.

\textbf{Position 1 ($y_1 = \SI{90}{m}$):}
Given constructive interference (in phase).
$\Delta x_1 = \sqrt{120^2 + 4(90)^2} - 120 = \sqrt{14400 + 32400} - 120 = \sqrt{46800} - 120$
$\Delta x_1 \approx 216.33 - 120 = 96.33 \text{ m}$.

\textbf{Position 2 ($y_2 = 90 + h$):}
The intensity decreases to zero and becomes maximum again. This corresponds to the path difference changing by one wavelength $\lambda$.
\[ \Delta x_2 - \Delta x_1 = \lambda \]
\[ (\sqrt{120^2 + 4(90+h)^2} - 120) - 96.33 = 1.33 \]
\[ \sqrt{14400 + 4(90+h)^2} = 120 + 96.33 + 1.33 = 217.66 \]
Squaring both sides:
\[ 14400 + 4(90+h)^2 = (217.66)^2 \approx 47376 \]
\[ 4(90+h)^2 = 47376 - 14400 = 32976 \]
\[ (90+h)^2 = 8244 \]
\[ 90+h = \sqrt{8244} \approx 90.8 \]
\[ h \approx 0.8 \text{ m} \]

\textbf{Answer:} $h \approx \mathbf{0.80 \text{ m}}$.
\newpage
\subsection*{(2b) Standing Wave - Sonometer Wire}
Given:
Wire density $\rho_w = 8.885 \times 10^3 \text{ kg m}^{-3}$, diameter $d_w = 0.51 \text{ mm}$.
Length $l = 0.60 \text{ m}$.
Cylinder diameter $D = 5.0 \text{ cm}$, height $H = 10.0 \text{ cm}$. Volume $V_c = \pi (\frac{D}{2})^2 H$.
Frequency formula for 2nd harmonic ($n=2$):
\[ f = \frac{2}{2l} \sqrt{\frac{T}{\mu}} = \frac{1}{l} \sqrt{\frac{T}{\mu}} \]
Mass per unit length of wire: $\mu = \rho_w A_w = \rho_w \pi (\frac{d_w}{2})^2$.
$\mu = 8885 \times \pi \times (0.255 \times 10^{-3})^2 \approx 8885 \times \pi \times 6.5 \times 10^{-8} \approx 1.815 \times 10^{-3} \text{ kg/m}$.

Let $T_1$ be tension when half immersed, $T_2$ when fully immersed.
$f_1 = 118.4 \text{ Hz}$, $f_2 = 114.7 \text{ Hz}$.
From frequency formula: $T = \mu (f l)^2$.
$T_1 = 1.815 \times 10^{-3} \times (118.4 \times 0.6)^2 = 1.815 \times 10^{-3} \times (71.04)^2 \approx 9.16 \text{ N}$.
$T_2 = 1.815 \times 10^{-3} \times (114.7 \times 0.6)^2 = 1.815 \times 10^{-3} \times (68.82)^2 \approx 8.60 \text{ N}$.

Forces on cylinder: Weight $W = \rho_c V_c g$, Buoyancy $U$.
$T_1 = W - \frac{1}{2} \rho_L V_c g$
$T_2 = W - \rho_L V_c g$
Subtracting equations:
$T_1 - T_2 = \frac{1}{2} \rho_L V_c g$.
$\rho_L = \frac{2(T_1 - T_2)}{V_c g}$.

Calculate $V_c$:
$V_c = \pi (0.025)^2 (0.1) = \pi (6.25 \times 10^{-4}) (0.1) \approx 1.963 \times 10^{-4} \text{ m}^3$.
Difference in Tension: $T_1 - T_2 = 9.16 - 8.60 = 0.56 \text{ N}$.
$\rho_L = \frac{2(0.56)}{(1.963 \times 10^{-4})(9.81)} = \frac{1.12}{1.926 \times 10^{-3}} \approx 581 \text{ kg m}^{-3}$.

Calculate $\rho_c$:
From $T_2 = \rho_c V_c g - \rho_L V_c g = (\rho_c - \rho_L) V_c g$.
$\rho_c - \rho_L = \frac{T_2}{V_c g} = \frac{8.60}{(1.963 \times 10^{-4})(9.81)} = \frac{8.60}{1.926 \times 10^{-3}} \approx 4465 \text{ kg m}^{-3}$.
$\rho_c = 4465 + 581 = 5046 \text{ kg m}^{-3}$.

\textbf{Answer:}
Density of cylinder $\approx \mathbf{5050 \text{ kg m}^{-3}}$.
Density of liquid $\approx \mathbf{580 \text{ kg m}^{-3}}$.

\newpage

% ==========================================
% QUESTION 3
% ==========================================
\section*{Question 3: Gravitation and Thermodynamics}

\subsection*{(3a) Gravitational Potential of Hollow Sphere}
Consider a uniform shell with inner radius $r$ and outer radius $R$. Density $\rho = \frac{M}{\frac{4}{3}\pi(R^3-r^3)}$.
Potential at distance $x$ from center ($V(\infty)=0$):
1. **Outer Surface ($x=R$):** Behaves as point mass $M$.
   \[ V_{out} = -\frac{GM}{R} \]
2. **Inner Surface ($x=r$):**
   Potential inside a cavity of a thick shell is constant and equal to the potential at the inner surface due to the entire mass distribution.
   Consider building the shell from thin shells of radius $s$ and thickness $ds$.
   Potential at center (and everywhere in cavity $r' < r$) is sum of potentials $dV = - \frac{G dm}{s}$.
   $dm = 4\pi s^2 \rho ds$.
   \[ V_{in} = \int_{r}^{R} -\frac{G (4\pi s^2 \rho ds)}{s} = -4\pi G \rho \int_{r}^{R} s ds \]
   \[ V_{in} = -4\pi G \rho \left[ \frac{s^2}{2} \right]_r^R = -2\pi G \rho (R^2 - r^2) \]
   Substitute $\rho = \frac{M}{\frac{4}{3}\pi(R^3-r^3)}$:
   \[ V_{in} = -2\pi G \left( \frac{3M}{4\pi(R^3-r^3)} \right) (R^2 - r^2) = -\frac{3GM}{2} \frac{(R-r)(R+r)}{(R-r)(R^2+Rr+r^2)} \]
   \[ V_{in} = -\frac{3GM}{2} \frac{R+r}{R^2+Rr+r^2} \]

   \textbf{Ratio:}
   \[ \frac{V_{out}}{V_{in}} = \frac{-GM/R}{-\frac{3GM}{2} \frac{R+r}{R^2+Rr+r^2}} = \frac{2}{3R} \frac{R^2+Rr+r^2}{R+r} \]
\newpage
\subsection*{(3b) Thermodynamics and Modern Physics}

\subsubsection*{3b(i) Entropy Change}
Copper rod: $L=0.30$ m, $d=0.02$ m. Area $A = \pi (0.01)^2 = 3.14 \times 10^{-4} \text{ m}^2$.
$T_H = 200^\circ\text{C} = 473 \text{ K}$, $T_C = 0^\circ\text{C} = 273 \text{ K}$.
Conductivity $k = 400 \text{ W m}^{-1} \text{K}^{-1}$.
Heat flow rate $\dot{Q}$:
\[ \dot{Q} = kA \frac{T_H - T_C}{L} = 400 (3.14 \times 10^{-4}) \frac{200}{0.30} \]
\[ \dot{Q} = \frac{25.12}{0.3} \approx 83.73 \text{ W} \]
Rate of entropy change of the universe (system):
\[ \dot{S}_{gen} = \frac{\dot{Q}}{T_C} - \frac{\dot{Q}}{T_H} = 83.73 \left( \frac{1}{273} - \frac{1}{473} \right) \]
\[ \dot{S}_{gen} = 83.73 (0.00366 - 0.00211) = 83.73 (0.00155) \approx \mathbf{0.130 \text{ J K}^{-1} \text{s}^{-1}} \]
\newpage
\subsubsection*{3b(ii) Hydrogen Collision}
Electron speed $v = 2.08 \times 10^6 \text{ m/s}$.
Kinetic Energy:
\[ K = \frac{1}{2} m_e v^2 = 0.5 (9.11 \times 10^{-31}) (2.08 \times 10^6)^2 \approx 1.97 \times 10^{-18} \text{ J} \]
In eV: $K = \frac{1.97 \times 10^{-18}}{1.60 \times 10^{-19}} \approx 12.3 \text{ eV}$.
Hydrogen energy levels: $E_1 = -13.6$, $E_2 = -3.4$, $E_3 = -1.51$ eV.
Excitation energies from ground state:
$\Delta E_{1 \to 2} = 10.2$ eV.
$\Delta E_{1 \to 3} = 12.09$ eV.
$\Delta E_{1 \to 4} = 12.75$ eV.
The electron ($12.3$ eV) can excite the atom to $n=2$ or $n=3$.
Possible de-excitation photons:
1. From $n=3 \to n=1$: $\lambda = \frac{1240}{12.09} \approx \mathbf{102.6 \text{ nm}}$.
2. From $n=3 \to n=2$: $\lambda = \frac{1240}{1.89} \approx \mathbf{656 \text{ nm}}$.
3. From $n=2 \to n=1$: $\lambda = \frac{1240}{10.2} \approx \mathbf{121.6 \text{ nm}}$.

\newpage

% ==========================================
% QUESTION 4
% ==========================================
\section*{Question 4: Electromagnetism}
\newpage
\subsection*{4(a) Oscillating Particle in Charged Ring}
Ring radius $a$, total charge $Q$. Field on axis at distance $x$:
\[ E_x = \frac{Qx}{4\pi\epsilon_0 (a^2+x^2)^{3/2}} \]
For $x \ll a$, $(a^2+x^2)^{3/2} \approx a^3$.
\[ E_x \approx \frac{Q}{4\pi\epsilon_0 a^3} x \]
Force on charge $-q$:
\[ F = -q E_x = - \left( \frac{qQ}{4\pi\epsilon_0 a^3} \right) x \]
This is a restoring force of the form $F = -kx$ with $k = \frac{qQ}{4\pi\epsilon_0 a^3}$.
Thus, the motion is Simple Harmonic Motion (SHM).
Frequency $f = \frac{1}{2\pi} \sqrt{\frac{k}{m}}$:
\[ f = \frac{1}{2\pi} \sqrt{\frac{qQ}{4\pi\epsilon_0 m a^3}} \]
Given linear charge density $\lambda$, $Q = 2\pi a \lambda$.
\[ f = \frac{1}{2\pi} \sqrt{\frac{q(2\pi a \lambda)}{4\pi\epsilon_0 m a^3}} = \frac{1}{2\pi} \sqrt{\frac{q\lambda}{2\epsilon_0 m a^2}} \]
\newpage
\subsection*{4(b) Magnetic Fields}
\subsubsection*{(i) Bohr Atom}
Electron orbit radius $r = 5.3 \times 10^{-10}$ m, speed $v = 6.91 \times 10^5$ m/s.
Equivalent current $I = \frac{q}{T} = \frac{e}{2\pi r / v} = \frac{ev}{2\pi r}$.
Magnetic field at center (proton):
\[ B = \frac{\mu_0 I}{2r} = \frac{\mu_0 e v}{4\pi r^2} \]
\[ B = \frac{(4\pi \times 10^{-7})(1.60 \times 10^{-19})(6.91 \times 10^5)}{4\pi (5.3 \times 10^{-10})^2} \]
\[ B = \frac{1.60 \times 6.91 \times 10^{-21}}{28.09 \times 10^{-20}} = \frac{11.056 \times 10^{-21}}{2.809 \times 10^{-19}} \approx 0.039 \text{ T} \]
\textbf{Answer:} $B \approx \mathbf{0.039 \text{ T}}$.

\subsubsection*{(ii) Spinning Disc}
Surface charge density $\sigma$, rotation speed $n$ rev/s ($\omega = 2\pi n$).
Consider a ring of radius $r$, width $dr$.
Charge $dq = \sigma (2\pi r dr)$.
Current $dI = dq \cdot n = \sigma 2\pi r n dr$.
Field at center due to ring: $dB = \frac{\mu_0 dI}{2r} = \frac{\mu_0 (\sigma 2\pi r n dr)}{2r} = \mu_0 \sigma \pi n dr$.
Total field:
\[ B = \int_0^R \mu_0 \sigma \pi n dr = \mu_0 \sigma \pi n R \]

\newpage

% ==========================================
% QUESTION 5
% ==========================================
\section*{Question 5: Special Relativity}

\subsection*{(a) Relative Velocity of Frame}
Event 1: $(x_1, t_1) = (0, 0)$.
Event 2: $(x_2, t_2) = (4c, 5)$. (Note: $4c$ denotes a distance of $4 \times 3 \times 10^8$ m).
In Frame $S'$, events are at the same spatial point $\implies \Delta x' = 0$.
Lorentz transform: $\Delta x' = \gamma (\Delta x - v \Delta t)$.
\[ 0 = \gamma (\Delta x - v \Delta t) \implies v = \frac{\Delta x}{\Delta t} \]
\[ v = \frac{4c}{5} = 0.8c \]
(i) Velocity of $S'$ relative to $S$ is $\mathbf{0.8c}$.

(ii) Time interval in $S'$:
\[ \Delta t' = \gamma (\Delta t - \frac{v \Delta x}{c^2}) \]
$\gamma = \frac{1}{\sqrt{1 - 0.8^2}} = \frac{1}{0.6} = \frac{5}{3}$.
\[ \Delta t' = \frac{5}{3} \left( 5 - \frac{(0.8c)(4c)}{c^2} \right) = \frac{5}{3} (5 - 3.2) = \frac{5}{3} (1.8) = 3 \text{ s} \]
\textbf{Answer:} $\mathbf{3 \text{ s}}$.

\subsection*{(b) Volume of Moving Cube}
Let $l$ be the rest length of the cube's sides.
Frame S: Cube moves with velocity $u$.
Observer moves with velocity $v$ in S.
The observer is in a frame $S'$ moving at $v$.
We need the velocity of the cube relative to the observer, $w$:
\[ w = \frac{u - v}{1 - \frac{uv}{c^2}} \]
The length of the cube in the direction of motion as measured by the observer is contracted by the Lorentz factor $\gamma_w = \frac{1}{\sqrt{1 - w^2/c^2}}$.
\[ l' = \frac{l}{\gamma_w} = l \sqrt{1 - \frac{w^2}{c^2}} \]
The transverse dimensions remain $l$.
Volume measured by observer:
\[ V' = l \cdot l \cdot l' = l^3 \sqrt{1 - \frac{w^2}{c^2}} \]
Substitute $w$:
\[ 1 - \frac{w^2}{c^2} = 1 - \frac{1}{c^2} \left( \frac{u-v}{1-uv/c^2} \right)^2 = \frac{(1-uv/c^2)^2 - (u/c - v/c)^2}{(1-uv/c^2)^2} \]
Let $\beta_u = u/c, \beta_v = v/c$.
Numerator: $(1 - \beta_u \beta_v)^2 - (\beta_u - \beta_v)^2 = (1 - \beta_u^2)(1 - \beta_v^2)$.
Therefore:
\[ V' = l^3 \frac{\sqrt{(1 - u^2/c^2)(1 - v^2/c^2)}}{1 - uv/c^2} \]

\end{document}