\documentclass[a4paper,12pt]{article}
\usepackage[utf8]{inputenc}
\usepackage[T1]{fontenc}
\usepackage{amsmath}
\usepackage{amssymb}
\usepackage{graphicx}
\usepackage{geometry}
\usepackage{siunitx}
\usepackage{fancyhdr}

% Page Setup
\geometry{margin=1in}
\pagestyle{fancy}
\fancyhf{}
\lhead{SPhO 2020 Theory Paper Solutions}
\rhead{\thepage}

\title{\textbf{33\textsuperscript{rd} Singapore Physics Olympiad (SPhO) 2020} \\ \Large Theory Paper Solutions}
\author{}
\date{}

\begin{document}

\maketitle
%\tableofcontents
%\newpage

% ==========================================
% QUESTION 1
% ==========================================
\section*{Question 1: Composite Rod Dynamics}

\subsection*{Problem Statement}
A cylindrical rod (radius $r=1.0$ cm, length $L=1.0$ m) consists of two sections of length $L/2 = 0.5$ m each. One section is zinc ($\rho_z = 7135 \text{ kg m}^{-3}$) and the other is copper ($\rho_c = 8940 \text{ kg m}^{-3}$). The zinc end is pivoted at O. The rod is released from horizontal. Determine the angular velocity when vertical.

\subsection*{Solution}
Let the rod rotate about point O (the end of the zinc section).
Total length $L = 1.0$ m. Radius $r = 0.01$ m. Cross-sectional area $A = \pi r^2$.
Volume of each section $V = A (L/2)$.

\textbf{1. Masses of the sections:}
\[ m_z = \rho_z V = \rho_z \pi r^2 \frac{L}{2} \]
\[ m_c = \rho_c V = \rho_c \pi r^2 \frac{L}{2} \]

\textbf{2. Center of Mass positions (from pivot O):}
The zinc section extends from $x=0$ to $x=L/2$. Its COM is at $x_z = L/4 = 0.25$ m.
The copper section extends from $x=L/2$ to $x=L$. Its COM is at $x_c = L/2 + L/4 = 0.75$ m.

\textbf{3. Moment of Inertia ($I$) about pivot O:}
Using the parallel axis theorem or integration:
For the Zinc section (pivoted at end):
\[ I_z = \frac{1}{3} m_z \left(\frac{L}{2}\right)^2 \]
For the Copper section (pivoted at distance $L/2$ from its start):
\[ I_c = I_{c, CM} + m_c x_c^2 = \frac{1}{12} m_c \left(\frac{L}{2}\right)^2 + m_c \left(\frac{3L}{4}\right)^2 \]
Total Moment of Inertia $I_{total} = I_z + I_c$.

\textbf{4. Conservation of Energy:}
Loss in Potential Energy ($\Delta PE$) = Gain in Rotational Kinetic Energy ($K_{rot}$).
When vertical, the center of mass of each section falls by a distance equal to its position $x_{CM}$.
\[ \Delta PE = m_z g x_z + m_c g x_c \]
\[ K_{rot} = \frac{1}{2} I_{total} \omega^2 \]
\[ \frac{1}{2} I_{total} \omega^2 = (m_z x_z + m_c x_c) g \]
\[ \omega = \sqrt{\frac{2 g (m_z x_z + m_c x_c)}{I_{total}}} \]

\textbf{Calculation:}
Common factor $k = \pi r^2 (L/2)$.
$m_z = 7135 k$, $m_c = 8940 k$.
$x_z = 0.25$, $x_c = 0.75$.
$I_z = \frac{1}{3} (7135 k) (0.5)^2 = 7135 k (0.0833)$.
$I_c = 8940 k \left[ \frac{1}{12}(0.5)^2 + (0.75)^2 \right] = 8940 k [0.02083 + 0.5625] = 8940 k (0.5833)$.

Substituting $k$ out (it cancels in the ratio):
Numerator term (Moment of mass): $M_{mom} = 7135(0.25) + 8940(0.75) = 1783.75 + 6705 = 8488.75$.
Denominator term (Inertia factor): $I_{fact} = \frac{7135}{3}(0.25) + 8940(0.5833) = 594.58 + 5214.7 = 5809.3$.
Wait, check $I_z$ formula: $L_{sec} = 0.5$. $I = m L_{sec}^2 / 3$. $0.5^2/3 = 0.0833$. Correct.
$I_{fact} = 7135(0.0833) + 8940(0.58333) = 594.58 + 5215 = 5809.6$.

\[ \omega = \sqrt{\frac{2(9.81)(8488.75)}{5809.6}} = \sqrt{\frac{166549}{5809.6}} = \sqrt{28.66} \approx 5.35 \text{ rad s}^{-1} \]

\textbf{Answer:} $\omega \approx 5.35 \text{ rad s}^{-1}$.

\newpage

% ==========================================
% QUESTION 2
% ==========================================
\section*{Question 2: Waves and Optics}

\subsection*{(a) Doppler Effect on Swing}
\textbf{Problem:} Swing $L=5$ m, amplitude $\theta_{max} = 45^\circ$. Source $f_s = 400$ Hz fixed near equilibrium position. Observer on swing. Speed of sound $c = 330$ m/s. Find max/min frequency heard.

\textbf{Solution:}
Max speed of swing $v_{max}$ occurs at the lowest point (equilibrium).
\[ v_{max} = \sqrt{2gL(1 - \cos\theta_{max})} \]
\[ v_{max} = \sqrt{2(9.81)(5)(1 - \cos 45^\circ)} = \sqrt{98.1 (1 - 0.707)} = \sqrt{98.1 (0.293)} \approx \sqrt{28.73} \approx 5.36 \text{ m/s} \]

The observer moves towards the source (at equilibrium) with $v_{max}$ and away with $v_{max}$.
Max Frequency (moving towards):
\[ f_{max} = f_s \left( \frac{c + v_{obs}}{c} \right) = 400 \left( \frac{330 + 5.36}{330} \right) = 400 (1.0162) \approx 406.5 \text{ Hz} \]
Min Frequency (moving away):
\[ f_{min} = f_s \left( \frac{c - v_{obs}}{c} \right) = 400 \left( \frac{330 - 5.36}{330} \right) = 400 (0.9837) \approx 393.5 \text{ Hz} \]

\textbf{Answer:} $f_{max} \approx 406.5$ Hz, $f_{min} \approx 393.5$ Hz.

\subsection*{(b) Compound Microscope}
\textbf{Problem:} $f_o = 6.0$ mm, $f_e = 40.0$ mm. Separation $L = 200$ mm. Final image virtual at $D = 250$ mm from eye lens.
(i) Object distance $u_o$. (ii) Magnifying power $M$.

\textbf{Solution:}
1. \textbf{Eyepiece:}
Image $v_e = -250$ mm (virtual). Focal length $f_e = 40$ mm.
$\frac{1}{v_e} - \frac{1}{u_e} = \frac{1}{f_e} \implies \frac{1}{-250} - \frac{1}{u_e} = \frac{1}{40}$
$\frac{1}{u_e} = -\frac{1}{250} - \frac{1}{40} = -\frac{4+25}{1000} = -\frac{29}{1000}$.
$u_e = -\frac{1000}{29} \approx -34.48$ mm. (Distance is 34.48 mm).

2. \textbf{Objective:}
The image formed by the objective acts as the object for the eyepiece.
Distance between lenses $L = v_o + |u_e| = 200$ mm.
$v_o = 200 - 34.48 = 165.52$ mm.
Focal length $f_o = 6.0$ mm.
$\frac{1}{v_o} - \frac{1}{u_o} = \frac{1}{f_o} \implies \frac{1}{u_o} = \frac{1}{v_o} - \frac{1}{f_o}$
$\frac{1}{u_o} = \frac{1}{165.52} - \frac{1}{6} \approx 0.00604 - 0.16667 = -0.1606$.
$u_o \approx -6.22$ mm.
(i) Object distance is \textbf{6.22 mm}.

3. \textbf{Magnifying Power ($M$):}
$M = M_o \times M_e = \left( \frac{v_o}{u_o} \right) \left( \frac{D}{u_e} \right)$? No, for final image at $D$:
$M \approx \frac{v_o}{u_o} \left( 1 + \frac{D}{f_e} \right)$.
$M_o = \frac{165.52}{6.22} \approx 26.6$.
$M_e = 1 + \frac{250}{40} = 1 + 6.25 = 7.25$.
$M = 26.6 \times 7.25 \approx 193$.

\textbf{Answer:} (i) 6.22 mm, (ii) $\approx 193$.

\newpage

% ==========================================
% QUESTION 3
% ==========================================
\section*{Question 3: Electromagnetic Induction}

\subsection*{Problem Statement}
Rod length $l$, resistance $R$, pivoted at one end, rotating with $\omega$ in B field (normal). Ends connected to resistor $R_0$.
(a) EMF induced. (b) Power in $R_0$. (c) Origin of power.

\subsection*{Solution}
\textbf{(a) Induced EMF:}
Consider an element $dr$ at distance $r$. Velocity $v = \omega r$.
$d\mathcal{E} = B v dr = B \omega r dr$.
\[ \mathcal{E} = \int_0^l B \omega r dr = \frac{1}{2} B \omega l^2 \]

\textbf{(b) Power in Resistor $R_0$:}
Total resistance of circuit $R_{tot} = R + R_0$ (assuming rod and resistor in series loop via pivot and rim).
Current $I = \frac{\mathcal{E}}{R + R_0}$.
Power developed in $R_0$:
\[ P_{R_0} = I^2 R_0 = \left( \frac{B \omega l^2}{2(R+R_0)} \right)^2 R_0 = \frac{B^2 \omega^2 l^4 R_0}{4(R+R_0)^2} \]

\textbf{(c) Origin of Power:}
The electric power originates from the \textbf{mechanical work done} by the external torque required to keep the rod rotating at constant angular velocity $\omega$.
Evidence: The magnetic force on the current element $dF = I dr B$ creates a retarding torque $d\tau = r dF = I B r dr$.
Total Torque $\tau = \int_0^l I B r dr = \frac{1}{2} I B l^2$.
Mechanical Power supplied $P_{mech} = \tau \omega = \frac{1}{2} \left( \frac{B \omega l^2}{2 R_{tot}} \right) B l^2 \omega = \frac{B^2 \omega^2 l^4}{4 R_{tot}}$.
Total electrical power dissipated $P_{elec} = I^2 R_{tot} = \frac{B^2 \omega^2 l^4}{4 R_{tot}}$.
Since $P_{mech} = P_{elec}$, the power comes from the external agent maintaining rotation against the magnetic braking torque.

\newpage

% ==========================================
% QUESTION 4
% ==========================================
\section*{Question 4: Quantum Physics}

\subsection*{Problem Statement}
Electron collides with H (ground). H excites. De-excitation emits 2 photons. One is $\lambda_1 = 656.3$ nm. Electron $\lambda_{dB}' = 1.915$ nm after collision.
(a) Wavelength of other photon? (b) Initial speed of electron?

\subsection*{Solution}
\textbf{(a) Other Photon Wavelength:}
The photon $\lambda_1 = 656.3$ nm corresponds to energy $E_1 = \frac{hc}{\lambda_1} = \frac{1240 \text{ eV nm}}{656.3} \approx 1.89$ eV.
In Hydrogen, energy levels are $E_n = -\frac{13.6}{n^2}$ eV.
Energy difference $\Delta E_{3 \to 2} = -1.51 - (-3.40) = 1.89$ eV.
This confirms the transition is $n=3 \to n=2$ (Balmer $\alpha$).
The atom must have been excited to at least $n=3$.
Since it emits *two* photons during de-excitation to ground ($n=1$), and one is $3 \to 2$, the other must be $2 \to 1$.
Transition $2 \to 1$:
$\Delta E_{2 \to 1} = E_2 - E_1 = -3.40 - (-13.6) = 10.2$ eV.
Wavelength $\lambda_2 = \frac{hc}{\Delta E} = \frac{1240}{10.2} \approx 121.6$ nm.

\textbf{(b) Initial Electron Speed:}
Energy Conservation: $K_{initial} = K_{final} + \Delta E_{atom}$.
Atom excitation energy $\Delta E_{1 \to 3} = E_3 - E_1 = -1.51 - (-13.6) = 12.09$ eV.
Final kinetic energy of electron $K_{final}$ from de Broglie wavelength:
$p' = \frac{h}{\lambda_{dB}'}$.
$K_{final} = \frac{p'^2}{2m_e} = \frac{h^2}{2m_e \lambda'^2}$.
Using $hc = 1240$ eV nm and $m_e c^2 = 511,000$ eV:
$K_{final} = \frac{(hc)^2}{2 (m_e c^2) \lambda'^2} = \frac{(1240)^2}{2(511000)(1.915)^2} = \frac{1.537 \times 10^6}{1.022 \times 10^6 \times 3.667} \approx 0.41$ eV.
$K_{initial} = 0.41 + 12.09 = 12.50$ eV.
Speed $v_i = \sqrt{\frac{2K_i}{m_e}} = c \sqrt{\frac{2 K_i}{m_e c^2}} = (3 \times 10^8) \sqrt{\frac{2(12.5)}{511000}} \approx 3 \times 10^8 \sqrt{4.89 \times 10^{-5}}$.
$v_i \approx 3 \times 10^8 \times 0.007 = 2.1 \times 10^6$ m/s.

\textbf{Answer:} (a) 121.6 nm, (b) $2.1 \times 10^6$ m/s.

\newpage

% ==========================================
% QUESTION 5
% ==========================================
\section*{Question 5: Projectile from Airplane}

\subsection*{Problem Statement}
Plane speed $v_p = 540$ km/h $= 150$ m/s. Height $h=1200$ m. Projectile fired horizontal speed $u$ relative to plane.
Vehicle speed $v_v = 40$ m/s same direction, distance $d$ ahead.
(a) Min $d$ to hit? (b) If $d = 5d_{min}$, speed at impact?

\subsection*{Solution}
\textbf{(a) Minimum $d$:}
Time of flight $t = \sqrt{\frac{2h}{g}} = \sqrt{\frac{2400}{9.81}} \approx 15.64$ s.
Absolute horizontal speed of projectile $v_{x} = v_p + u$ (assuming fired forward).
If fired backward, $v_x = v_p - u$.
Condition for hit: Horizontal distance covered by projectile = Initial distance $d$ + Distance covered by vehicle.
$R_p = d + v_v t \implies (v_p + u) t = d + v_v t$.
$d = (v_p + u - v_v) t$.
Since $u$ is given as a variable "speed of u", and the question asks for "minimum value of d", this phrasing is tricky.
Interpretation 1: $u$ is variable, can be chosen. But usually projectiles are fired forward. If $u$ can be negative (fired backward relative to plane), $d$ can be smaller.
Interpretation 2: $u$ is a missing number in the text. Assuming typical relative speed, say $u=0$ (dropped) or $u$ is positive.
Let's assume $u$ is a fixed parameter given in the problem context (perhaps missing from text).
Expression: $d_{min} = (150 + u - 40) \times 15.64 = 110(15.64) + 15.64 u$.
If $u$ represents a speed relative to plane, the minimum $d$ generally corresponds to the minimum physically possible relative speed, e.g., $u=0$ (dropped). Or perhaps $d$ depends on $u$.
Without the numeric value of $u$, the answer is symbolic: $d = (110 + u) \sqrt{2h/g}$.

\textbf{(b) Impact Speed if $d = 5 d_{min}$:}
This implies $d$ is a variable depending on launch conditions, or $d_{min}$ is a specific reference.
If $d' = 5 d_{min}$, it implies the launch speed $u'$ must be different to hit, or the projectile misses.
Since the question asks "what is the speed... when it hits", it implies a hit occurs.
Thus, $u$ must be different.
$5 d_{min} = (150 + u' - 40) t$.
$5 (110 + u_{min}) t = (110 + u') t \implies 550 + 5u_{min} = 110 + u'$.
$u' = 440 + 5 u_{min}$.
Final speed $v_f = \sqrt{v_{x}'^2 + v_y^2}$.
$v_y = g t = \sqrt{2gh}$.
$v_x' = 150 + u'$.
$v_f = \sqrt{(150 + 440 + 5u_{min})^2 + 2gh}$.

\textbf{Assumption:} If the "minimum d" corresponds to $u=0$ (just dropped):
$d_{min} = 110 \times 15.64 = 1720$ m.
For part (b), $d = 5(1720) = 8600$.
$8600 = (110 + u') 15.64 \implies 110 + u' = 550 \implies u' = 440$ m/s.
$v_x' = 150 + 440 = 590$ m/s.
$v_y = 9.81 \times 15.64 = 153.4$ m/s.
$v_{impact} = \sqrt{590^2 + 153.4^2} \approx 610$ m/s.

\newpage

% ==========================================
% QUESTION 6
% ==========================================
\section*{Question 6: Satellite Photos}

\subsection*{Problem Statement}
Satellite $h=400$ km. Equator orbit (sense follows Earth). Photos of region P on equator. How many in 24h?

\subsection*{Solution}
Radius of orbit $r = R_E + h = 6400 + 400 = 6800$ km $= 6.8 \times 10^6$ m.
Orbital velocity $v = \sqrt{GM/r} = \sqrt{g R_E^2 / r}$.
Period $T_{sat} = 2\pi \sqrt{r^3/GM}$.
Using $g=9.81$, $R_E=6.37 \times 10^6$ (standard):
$T_{sat} = 2\pi \sqrt{\frac{(6.8 \times 10^6)^3}{9.81 (6.37 \times 10^6)^2}} \approx 5550 \text{ s} \approx 92.5$ min.
Angular velocity of satellite $\omega_s = \frac{2\pi}{T_{sat}}$.
Angular velocity of Earth $\omega_E = \frac{2\pi}{24 \times 3600}$.
Relative angular velocity $\omega_{rel} = \omega_s - \omega_E$ (since moving in same direction).
Time between encounters $T_{rel} = \frac{2\pi}{\omega_{rel}} = \frac{2\pi}{2\pi/T_s - 2\pi/T_E} = \frac{1}{1/T_s - 1/T_E}$.
$T_{rel} = \frac{T_s T_E}{T_E - T_s}$.
Using $T_E = 1440$ min, $T_s = 92.5$ min.
$T_{rel} = \frac{92.5 \times 1440}{1440 - 92.5} = \frac{133200}{1347.5} \approx 98.8$ min.
Number of photos in 24h ($T_{total} = 1440$ min):
$N = \frac{1440}{98.8} \approx 14.57$.
The satellite passes P roughly 14 or 15 times.
Number of photos = 14 (integer count).

\textbf{Answer:} 14 photos.

\newpage

% ==========================================
% QUESTION 7 (a)
% ==========================================
\section*{Question 7: Electrostatics in Tunnel}

\subsection*{Problem Statement}
Sphere radius $R$, charge density $\rho$. Tunnel through diameter. Particle $m, -q$.
(a) Motion? Speed at center?

\subsection*{Solution}
\textbf{Motion:}
Electric field inside uniform sphere at radius $r$:
$E(r) = \frac{\rho r}{3\epsilon_0}$.
Force on particle $-q$: $F = -q E = -\frac{q\rho}{3\epsilon_0} r$.
This is a restoring force $F = -k r$ with $k = \frac{q\rho}{3\epsilon_0}$.
The motion is \textbf{Simple Harmonic Motion} about the center.

\textbf{Speed at Center:}
Conservation of Energy. Amplitude $A = R$ (released at surface).
Maximum speed $v_{max} = \omega A$.
Angular frequency $\omega = \sqrt{\frac{k}{m}} = \sqrt{\frac{q\rho}{3m\epsilon_0}}$.
$v_{center} = R \sqrt{\frac{q\rho}{3m\epsilon_0}}$.

\newpage

% ==========================================
% QUESTION 7 (b)
% ==========================================
\section*{Question 7(b): Particle under Forces}

\subsection*{Problem Statement}
Particle between A and B ($L$). Force towards A: $F_A = \lambda r_A$. Force towards B: $F_B = \lambda r_B$.
(i) Show SHM. Period? (ii) Amplitude if rest at midpoint?

\subsection*{Solution}
Let origin be at A. Position $x$.
$r_A = x$. $r_B = L - x$.
Net Force (towards right positive):
$F_{net} = -F_A + F_B = -\lambda x + \lambda (L - x) = \lambda L - 2\lambda x$.
Equilibrium position: $F_{net} = 0 \implies 2\lambda x = \lambda L \implies x = L/2$.
Let displacement from equilibrium $y = x - L/2$.
$F_{net} = \lambda L - 2\lambda (y + L/2) = \lambda L - 2\lambda y - \lambda L = -2\lambda y$.
Since force is proportional to displacement $-y$, the motion is SHM.

\textbf{Period:}
Effective spring constant $k_{eff} = 2\lambda$.
$T = 2\pi \sqrt{\frac{m}{k_{eff}}} = 2\pi \sqrt{\frac{m}{2\lambda}}$.

\textbf{Amplitude/Energy:}
Snippet says "instantaneously at rest at the mid-point". Since the midpoint $L/2$ is the equilibrium position, if it is at rest there, it stays there. Amplitude = 0.
However, if the question implies "mid-point of oscillation" is somewhere else? No.
Maybe the "mid-point" refers to the midpoint of the *path* which is the equilibrium.
If the particle is at rest at the equilibrium, $E=0$.
Likely interpretation: The particle is *released* from rest at some point (maybe A or B?).
If the snippet means "instantaneously at rest at [End Point]", and equilibrium is midpoint.
Assuming amplitude $A$.
Kinetic Energy at zero force point (equilibrium) is $K_{max} = \frac{1}{2} m \omega^2 A^2 = \frac{1}{2} (2\lambda) A^2 = \lambda A^2$.

\newpage

% ==========================================
% QUESTION 8 (a)
% ==========================================
\section*{Question 8(a): Gas Expansion}

\subsection*{Problem Statement}
Ideal gas. $V_1, P_1, T_1$. Lid opened $\to$ Adiabatic expansion to $P_{atm}$. Lid closed. Warms to $T_1$. Final pressure $P_2$.
Find (i) Volume left at initial conditions? (ii) $\gamma$. (iii) Atomicity.

\subsection*{Solution}
Let standard atmospheric pressure be $P_a$.
\textbf{(ii) Ratio $\gamma$:}
Initial state: $P_1, V_1, T_1$.
Adiabatic expansion: The gas *remaining* in the container (volume $V_1$) was originally a smaller volume $V'$ in the initial state.
It expands from $P_1$ to $P_a$ adiabatically.
Relation: $P_1 (V')^\gamma = P_a (V_1)^\gamma \implies \frac{V'}{V_1} = \left(\frac{P_a}{P_1}\right)^{1/\gamma}$.
Temperature after expansion $T'$: $T' = T_1 \left(\frac{P_a}{P_1}\right)^{(\gamma-1)/\gamma}$.
Final state (after warming): Volume $V_1$, Temp $T_1$. Pressure $P_2$.
For the gas mass inside: $\frac{P_a V_1}{T'} = \frac{P_2 V_1}{T_1} \implies P_2 = P_a \frac{T_1}{T'}$.
Substitute $T'$: $P_2 = P_a \frac{T_1}{T_1 (P_a/P_1)^{(\gamma-1)/\gamma}} = P_a \left(\frac{P_1}{P_a}\right)^{(\gamma-1)/\gamma}$.
$\ln P_2 = \ln P_a + \frac{\gamma-1}{\gamma} (\ln P_1 - \ln P_a)$.
$\ln(P_2/P_a) = (1 - 1/\gamma) \ln(P_1/P_a)$.
$\frac{\ln P_2 - \ln P_a}{\ln P_1 - \ln P_a} = 1 - \frac{1}{\gamma}$.
$\frac{1}{\gamma} = 1 - \frac{\ln(P_2/P_a)}{\ln(P_1/P_a)} = \frac{\ln P_1 - \ln P_a - \ln P_2 + \ln P_a}{\ln P_1 - \ln P_a} = \frac{\ln P_1 - \ln P_2}{\ln P_1 - \ln P_a}$.
$\gamma = \frac{\ln(P_1/P_a)}{\ln(P_1/P_2)}$.

\textbf{(i) Volume of gas left (at initial conditions):}
The mass left corresponds to volume $V'$ at $P_1, T_1$.
$V' = V_1 (P_a/P_1)^{1/\gamma} = V_1 (P_a/P_1)^{\frac{\ln(P_1/P_2)}{\ln(P_1/P_a)}} = V_1 \exp\left( \ln(P_a/P_1) \cdot \frac{\ln(P_1/P_2)}{\ln(P_1/P_a)} \right)$.
$V' = V_1 \exp(-\ln(P_1/P_2)) = V_1 (P_2/P_1)$.
Wait. Check: $P_1 V' = P_2 V_1$ (Isothermal relation for same mass).
Since final state is $P_2, V_1, T_1$ and effective initial state of that mass was $P_1, V', T_1$. Yes, $P_1 V' = P_2 V_1$.
So $V' = V_1 (P_2/P_1)$.

\textbf{(iii) Atomicity:}
Calculate $\gamma$ from given P values (missing in text, but formula provided).
If $\gamma \approx 1.67$, monatomic. If $1.4$, diatomic.

\newpage

% ==========================================
% QUESTION 8 (b)
% ==========================================
\section*{Question 8(b): Thermal Conduction}

\subsection*{Problem Statement}
Cu rod ($D=2$ cm, $L=50$ cm). Hot end $T_H$. Cold end in ice/water ($0^\circ$C). Time to rise from $T_A$ to $T_B$?
(Assuming rise refers to the melted water mixture or implies melting time).

\subsection*{Solution}
Heat Current $\dot{Q} = k A \frac{T_H - T_{mix}}{L}$.
Usually $T_{mix} \approx 0$ during melting.
Heat required $Q = m_{ice} L_f + m_{total} c_w \Delta T$.
Time $t = \int \frac{dQ}{\dot{Q}}$.
If $T_{mix}$ changes, $\dot{Q}$ changes.
$\frac{d Q}{dt} = (m_i+m_w) c_w \frac{dT}{dt} = k A \frac{T_H - T}{L}$.
$\int_{T_A}^{T_B} \frac{dT}{T_H - T} = \int_0^t \frac{kA}{L M c} dt$.
$-\ln(T_H - T) |_{T_A}^{T_B} = \frac{kA}{LMc} t$.
$t = \frac{L M c}{k A} \ln\left(\frac{T_H - T_A}{T_H - T_B}\right)$.
where $M = 0.3$ kg (total mass).

\end{document}