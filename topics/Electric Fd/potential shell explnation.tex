\documentclass{article}
\usepackage{amsmath}
\usepackage{geometry}
\geometry{a4paper, margin=1in}
\title{Reconciling Potential Definition with Calculation Method}
\author{A Detailed Explanation}
\date{}

\begin{document}
\maketitle

\section*{The Core Question}

You have raised an excellent and insightful point. Your understanding is perfectly correct: the definition of gravitational potential at a point is the work done per unit mass in bringing a test mass from infinity to that point. How, then, can we justify calculating the potential at the center of a hollow sphere by integrating from the inner radius $r$ to the outer radius $R$?

The answer lies in the powerful \textbf{Principle of Superposition}.

\section{The Principle of Superposition}

Gravitational potential is a scalar quantity. This is a crucial property. It means that the total potential at a point, created by a large object, is simply the algebraic sum of the potentials created by all its individual parts.
\[
V_{\text{total}} = V_1 + V_2 + V_3 + \dots
\]
For a continuous object like our thick hollow sphere, this sum becomes an integral. We can conceptually break down the thick sphere into an infinite number of infinitesimally thin, concentric shells. The total potential at the center is the sum (integral) of the potential contributions from each of these thin shells.
\[
V_{\text{total at center}} = \int dV_{\text{shell at center}}
\]

\section{Potential from a Single Thin Shell}

Let's focus on a single, infinitesimally thin shell of mass $dM$ and radius $x'$. Your understanding of "work done from infinity" is precisely how we find the potential contribution from this \textit{one} thin shell.

According to Newton's Shell Theorem:
\begin{enumerate}
    \item The gravitational field \textit{inside} a thin spherical shell is zero.
    \item This means no work is done to move a test mass around inside the shell.
    \item Therefore, the gravitational potential is \textbf{constant} everywhere inside that thin shell.
\end{enumerate}

The value of this constant potential inside the shell is equal to the potential at its surface. To find the potential at the surface, we use your definition: we calculate the work done to bring a unit mass from infinity to the surface of this thin shell. Because we are always outside the shell until we reach its surface, it behaves gravitationally like a point mass $dM$ at its center.

So, the potential at the surface of our single thin shell (and everywhere inside it, including at the very center of the system) due to the shell itself is:
\[
dV_{\text{shell at center}} = -\frac{G \, dM}{x'}
\]
Here, $x'$ is the radius of that specific thin shell. This formula is a direct result of applying the "work done from infinity" concept to a single thin shell.

\section{Summing It All Up (The Integration)}

Now we return to the principle of superposition. The total potential at the center of the entire hollow sphere is the sum of the potentials from \textit{all} the thin shells that make it up.

Our hollow sphere is composed of these shells, with radii ranging from $x' = r$ all the way out to $x' = R$. We need to add up the potential contribution, $dV_{\text{shell at center}}$, for every single one of these shells. This "summing up" of an infinite number of infinitesimal parts is precisely what an integral does.

We substitute our expression for the potential from one shell into the superposition integral:
\[
V_{\text{total at center}} = \int_{\text{all shells}} dV_{\text{shell at center}} = \int_{r}^{R} \left(-\frac{G \, dM}{x'}\right)
\]
As derived previously, the mass of a shell of radius $x'$ and thickness $dx'$ is $dM = 4\pi\rho (x')^2 dx'$. Substituting this in gives the integral we used in the solution:
\[
V_{\text{total at center}} = \int_{r}^{R} -G \frac{4\pi\rho (x')^2 dx'}{x'} = \int_{r}^{R} -4\pi G\rho x' dx'
\]

\section{Conclusion: Bridging the Two Ideas}

So, to be perfectly clear, the integral from $r$ to $R$ is \textbf{not} calculating the work done to move a mass from $r$ to $R$. Instead:
\begin{enumerate}
    \item You correctly understand that potential is defined by the work done from infinity.
    \item We use this definition to find the potential contribution from a \textbf{single, thin shell} ($dV = -G dM/x'$).
    \item We then use the \textbf{principle of superposition} to add up the potentials from all the thin shells that constitute the thick hollow sphere. This addition is performed by the integral from the inner radius ($r$) to the outer radius ($R$).
\end{enumerate}

In essence, the integral is a summation tool, not a work-done calculation over the path of integration. It sums up the individual "work-done-from-infinity" potentials of each infinitesimally thin layer that makes up the object.

\end{document}