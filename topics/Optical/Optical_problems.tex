\documentclass[12pt, a4paper]{article}

\usepackage{amsmath}
\usepackage{geometry}
\usepackage{tcolorbox}
\usepackage{siunitx}
\usepackage{graphicx}

\geometry{a4paper, margin=1in}
\linespread{1.3}

\title{\textbf{New Problems in Optics: Microscopes \& Telescopes}}
\author{Generated by Gemini}
\date{\today}

\begin{document}

\maketitle

\section*{New Problem 1: The Laboratory Microscope \(\text{\normalsize \reflectbox{}}\)}

\begin{tcolorbox}[colback=blue!5!white, colframe=blue!75!black, title=Question 1]
A biologist is using a compound microscope with an objective lens of focal length \SI{5.0}{\milli\meter} and an eyepiece of focal length \SI{25.0}{\milli\meter}. The lenses are \SI{18.0}{\centi\meter} apart. A specimen is placed under the objective lens, and the final virtual image is formed at the near point of the eye (\SI{25}{\centi\meter} from the eyepiece).

Calculate:
\begin{itemize}
    \item[(i)] The position of the intermediate image formed by the objective lens.
    \item[(ii)] The distance of the specimen from the objective lens.
    \item[(iii)] The total magnifying power of the microscope.
\end{itemize}
\end{tcolorbox}

\subsection*{Explanation and Solution}
\textbf{Given values:}
\begin{itemize}
    \item Objective focal length, \(f_o = \SI{5.0}{\milli\meter} = \SI{0.5}{\centi\meter}\).
    \item Eyepiece focal length, \(f_e = \SI{25.0}{\milli\meter} = \SI{2.5}{\centi\meter}\).
    \item Lens separation, \(L = \SI{18.0}{\centi\meter}\).
    \item Final image distance from eyepiece, \(v_e = \SI{-25}{\centi\meter}\) (negative for a virtual image).
    \item Near point, \(D = \SI{25}{\centi\meter}\).
\end{itemize}

\begin{enumerate}
    \item \textbf{Analyze the Eyepiece:} First, we find the position of the intermediate image (\(u_e\)), which acts as the object for the eyepiece. Using the thin lens formula:
    \[ \frac{1}{f_e} = \frac{1}{u_e} + \frac{1}{v_e} \]
    \[ \frac{1}{2.5} = \frac{1}{u_e} + \frac{1}{-25} \]
    \[ \frac{1}{u_e} = \frac{1}{2.5} + \frac{1}{25} = \frac{10}{25} + \frac{1}{25} = \frac{11}{25} \]
    \[ u_e = \frac{25}{11} \approx \SI{2.27}{\centi\meter} \]

    \item \textbf{Find the Objective's Image Distance (\(v_o\)):} The lens separation (\(L\)) is the sum of the objective's image distance (\(v_o\)) and the eyepiece's object distance (\(u_e\)).
    \[ L = v_o + u_e \]
    \[ 18.0 = v_o + 2.27 \]
    \[ v_o = 18.0 - 2.27 = \SI{15.73}{\centi\meter} \]
    \textbf{Answer (i): The intermediate image is formed \SI{15.73}{\centi\meter} from the objective lens.}

    \item \textbf{Analyze the Objective Lens:} Now we find the original position of the specimen (\(u_o\)) from the objective lens.
    \[ \frac{1}{f_o} = \frac{1}{u_o} + \frac{1}{v_o} \]
    \[ \frac{1}{0.5} = \frac{1}{u_o} + \frac{1}{15.73} \]
    \[ \frac{1}{u_o} = 2 - \frac{1}{15.73} \approx 2 - 0.0636 = 1.9364 \]
    \[ u_o = \frac{1}{1.9364} \approx \SI{0.516}{\centi\meter} \]
    \textbf{Answer (ii): The specimen is \SI{0.516}{\centi\meter} (or \SI{5.16}{\milli\meter}) from the objective lens.}

    \item \textbf{Calculate Magnifying Power:} The total magnifying power (\(M\)) is the product of the objective's linear magnification (\(m_o\)) and the eyepiece's angular magnification (\(M_e\)).
    \[ M = m_o \times M_e = \left(\frac{v_o}{u_o}\right) \times \left(\frac{D}{u_e}\right) \]
    \[ M = \left(\frac{15.73}{0.516}\right) \times \left(\frac{25}{2.27}\right) \]
    \[ M \approx (30.48) \times (11.01) \approx 335.6 \]
    \textbf{Answer (iii): The total magnifying power is approximately 336.}
\end{enumerate}

\newpage

\section*{New Problem 2: The Amateur's Telescope \(\text{\normalsize ��}\)}
\begin{tcolorbox}[colback=red!5!white, colframe=red!75!black, title=Question 2]
An amateur astronomer has a telescope that, in normal adjustment, has a magnifying power of 50. The objective lens has a focal length of \SI{100}{\centi\meter}.

\begin{itemize}
    \item[(i)] Calculate the focal length of the eyepiece and the separation between the lenses in normal adjustment.
    \item[(ii)] If the astronomer adjusts the eyepiece to view the final image at their near point (\SI{25}{\centi\meter} away), what is the new separation between the lenses?
\end{itemize}
\end{tcolorbox}

\subsection*{Explanation and Solution}
\textbf{Given values:}
\begin{itemize}
    \item Magnifying power (normal adjustment), \(M = 50\).
    \item Objective focal length, \(f_o = \SI{100}{\centi\meter}\).
    \item Near point, \(D = \SI{25}{\centi\meter}\).
\end{itemize}

\begin{enumerate}
    \item \textbf{Normal Adjustment Calculations:} In normal adjustment (final image at infinity), the magnifying power is given by:
    \[ M = \frac{f_o}{f_e} \]
    \[ 50 = \frac{100}{f_e} \implies f_e = \frac{100}{50} = \SI{2.0}{\centi\meter} \]
    The lens separation (\(L_1\)) in normal adjustment is the sum of the focal lengths:
    \[ L_1 = f_o + f_e \]
    \[ L_1 = 100 + 2.0 = \SI{102.0}{\centi\meter} \]
    \textbf{Answer (i): The eyepiece focal length is \SI{2.0}{\centi\meter}, and the lens separation is \SI{102}{\centi\meter}.}

    \item \textbf{Adjusted Position Calculations:} When the final image is formed at the near point, \(v_e = \SI{-25}{\centi\meter}\). We find the new object distance for the eyepiece (\(u_e\)).
    \[ \frac{1}{f_e} = \frac{1}{u_e} + \frac{1}{v_e} \]
    \[ \frac{1}{2.0} = \frac{1}{u_e} + \frac{1}{-25} \]
    \[ \frac{1}{u_e} = \frac{1}{2.0} + \frac{1}{25} = 0.5 + 0.04 = 0.54 \]
    \[ u_e = \frac{1}{0.54} \approx \SI{1.85}{\centi\meter} \]
    The intermediate image position (\(v_o\)) is still at the focal point of the objective, so \(v_o = f_o = \SI{100}{\centi\meter}\). The new lens separation (\(L_2\)) is:
    \[ L_2 = f_o + u_e \]
    \[ L_2 = 100 + 1.85 = \SI{101.85}{\centi\meter} \]
    \textbf{Answer (ii): The new separation between the lenses is \SI{101.85}{\centi\meter}.}
\end{enumerate}

\end{document}