\documentclass{article}
\usepackage[a4paper, margin=1in]{geometry}
\usepackage{amsmath}
\usepackage{amsfonts}
\usepackage{amssymb}

% Define a new environment for problems to keep them distinct
\newenvironment{problem}[1]
  {\begin{center}\bfseries #1\end{center}\par\nobreak\vspace{\medskipamount}}
  {\par\vspace{\bigskipamount}\hrule\vspace{\bigskipamount}}

\begin{document}

\begin{problem}{Problem 1: Compound Microscope - Magnification and Image Position}
\subsection*{Question}
A compound microscope has an \textbf{objective lens} with a focal length of \textbf{10 mm} and an \textbf{eyepiece} with a focal length of \textbf{40 mm}. An object is placed \textbf{12 mm} from the objective lens. The lenses are \textbf{200 mm} apart.
\begin{itemize}
    \item[a)] Where is the intermediate image formed by the objective lens?
    \item[b)] What is the linear magnification of the objective?
    \item[c)] What is the total magnification of the microscope, assuming the final image is viewed at the near point (\(D = 250\) mm)?
\end{itemize}

\subsection*{Solution}
\subsubsection*{Given Information}
\begin{itemize}
    \item Objective focal length, \(f_o = 10 \text{ mm}\)
    \item Eyepiece focal length, \(f_e = 40 \text{ mm}\)
    \item Object distance from objective, \(u_o = 12 \text{ mm}\)
    \item Distance between lenses, \(L = 200 \text{ mm}\)
    \item Near point distance, \(D = 250 \text{ mm}\)
\end{itemize}

\subsubsection*{a) Position of the Intermediate Image (\(v_o\))}
We use the thin lens formula for the objective lens. By convention, the distance to a real object is negative.
$$ \frac{1}{f_o} = \frac{1}{v_o} - \frac{1}{u_o} $$
Substituting the values, with \(u_o = -12 \text{ mm}\):
$$ \frac{1}{10} = \frac{1}{v_o} - \frac{1}{-12} \implies \frac{1}{10} = \frac{1}{v_o} + \frac{1}{12} $$
Rearranging to solve for \(v_o\):
$$ \frac{1}{v_o} = \frac{1}{10} - \frac{1}{12} = \frac{6 - 5}{60} = \frac{1}{60} $$
$$ v_o = 60 \text{ mm} $$
The intermediate image is formed \textbf{60 mm} from the objective lens, inside the tube.

\subsubsection*{b) Linear Magnification of the Objective (\(m_o\))}
The linear magnification of the objective is given by:
$$ m_o = -\frac{v_o}{u_o} $$
$$ m_o = -\frac{60}{-12} = 5 $$
The objective lens produces a real, inverted image magnified \textbf{5} times.

\subsubsection*{c) Total Magnification of the Microscope (\(M_{total}\))}
The eyepiece acts as a simple magnifier. Its angular magnification \(M_e\) for an image formed at the near point is:
$$ M_e = 1 + \frac{D}{f_e} $$
$$ M_e = 1 + \frac{250}{40} = 1 + 6.25 = 7.25 $$
The total magnification is the product of the objective and eyepiece magnifications:
$$ M_{total} = m_o \times M_e $$
$$ M_{total} = 5 \times 7.25 = 36.25 $$
The total magnification of the microscope is \textbf{36.25x}.
\end{problem}

\begin{problem}{Problem 2: Compound Microscope - Determining Focal Length}
\subsection*{Question}
A lab technician uses a compound microscope to view a sample. The eyepiece has a focal length of \textbf{25 mm}. The distance between the objective lens and the eyepiece is adjusted to \textbf{18 cm}. If the microscope provides a total magnification of \textbf{-250} when the final image is formed at infinity (normal adjustment), what is the focal length of the objective lens?

\subsection*{Solution}
\subsubsection*{Given Information}
\begin{itemize}
    \item Eyepiece focal length, \(f_e = 25 \text{ mm}\)
    \item Distance between lenses (approximated as tube length), \(L \approx 18 \text{ cm} = 180 \text{ mm}\)
    \item Total magnification, \(M_{total} = -250\)
    \item Near point distance, \(D = 25 \text{ cm} = 250 \text{ mm}\)
    \item The final image is at infinity (normal adjustment).
\end{itemize}

\subsubsection*{Calculation}
For a compound microscope in normal adjustment, the total magnification is given by the formula:
$$ M_{total} = \left( -\frac{L}{f_o} \right) \times \left( \frac{D}{f_e} \right) $$
This formula provides a good approximation where \(L\) is the effective tube length. We are given all values except for \(f_o\), so we rearrange the formula to solve for it.
$$ -250 = \left( -\frac{180}{f_o} \right) \times \left( \frac{250}{25} \right) $$
First, simplify the eyepiece magnification term:
$$ \frac{250}{25} = 10 $$
Now substitute this back into the main equation:
$$ -250 = -\frac{180}{f_o} \times 10 $$
The negative signs on both sides cancel out:
$$ 250 = \frac{1800}{f_o} $$
Finally, solve for \(f_o\):
$$ f_o = \frac{1800}{250} = \frac{180}{25} = 7.2 $$
The focal length of the objective lens is \textbf{7.2 mm}.
\end{problem}

\begin{problem}{Problem 3: Two-Lens Telescope - Angular Magnification and Length}
\subsection*{Question}
An astronomical telescope consisting of two converging lenses is used to view a distant galaxy. The \textbf{objective lens} has a focal length of \textbf{1500 mm}, and the \textbf{eyepiece} has a focal length of \textbf{30 mm}. The telescope is set for a relaxed eye (normal adjustment, with the final image at infinity).
\begin{itemize}
    \item[a)] What is the angular magnification of the telescope?
    \item[b)] What is the total length of the telescope (the distance between the two lenses)?
\end{itemize}

\subsection*{Solution}
\subsubsection*{Given Information}
\begin{itemize}
    \item Objective lens focal length, \(f_o = 1500 \text{ mm}\)
    \item Eyepiece focal length, \(f_e = 30 \text{ mm}\)
    \item The telescope is in normal adjustment.
\end{itemize}

\subsubsection*{a) Angular Magnification (\(M\))}
For a refracting telescope in normal adjustment, the angular magnification is the ratio of the objective's focal length to the eyepiece's focal length. The negative sign indicates an inverted image.
$$ M = -\frac{f_o}{f_e} $$
Plugging in the given values:
$$ M = -\frac{1500 \text{ mm}}{30 \text{ mm}} = -50 $$
The angular magnification is \textbf{-50x}.

\subsubsection*{b) Total Length of the Telescope (\(L\))}
In normal adjustment, a distant object (at infinity) forms an intermediate image at the objective's focal point (\(f_o\)). For the final image to also be at infinity, this intermediate image must be located at the eyepiece's focal point (\(f_e\)). Therefore, the two focal points must coincide, and the total length of the telescope is the sum of the two focal lengths.
$$ L = f_o + f_e $$
Plugging in the values:
$$ L = 1500 \text{ mm} + 30 \text{ mm} = 1530 \text{ mm} $$
The total length of the telescope is \textbf{1530 mm} (or 1.53 meters).
\end{problem}

\end{document}