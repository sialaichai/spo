\documentclass[12pt]{article}
\usepackage{amsmath}
\usepackage{graphicx}
\usepackage[a4paper, margin=1in]{geometry}

\title{Lesson Notes: Doppler Effect \& Beats}
\author{A 1-Hour Teaching Guide}
\date{}

\begin{document}
\maketitle

\section{The Doppler Effect: What is it? \quad \Large }

The \textbf{Doppler effect} is the change in the observed frequency of a wave when the \textbf{source} of the wave and the \textbf{observer} are moving relative to each other.

Think about an ambulance siren. As it comes towards you, the pitch sounds higher. As it moves away, the pitch sounds lower. The siren itself isn't changing its sound; your perception of its frequency changes due to the relative motion.

\begin{itemize}
    \item \textbf{Approaching:} Wavefronts get compressed. This means a shorter wavelength ($\lambda$) and thus a \textbf{higher observed frequency} ($f_o$).
    \item \textbf{Receding:} Wavefronts get stretched out. This means a longer wavelength ($\lambda$) and thus a \textbf{lower observed frequency} ($f_o$).
\end{itemize}

% To compile, you need an image file named 'doppler_waves.png' in the same directory.
% \begin{figure}[h!]
% \centering
% \includegraphics[width=0.8\textwidth]{doppler_waves.png} 
% \caption{Wavefronts compressing (approaching) and stretching (receding).}
% \end{figure}

\subsection{The General Equation}
The key to solving all Doppler effect problems is one master equation:
\[
f_o = f_s \left( \frac{v \pm v_o}{v \mp v_s} \right)
\]
Where:
\begin{itemize}
    \item $f_o$ = Observed frequency (what the listener hears)
    \item $f_s$ = Source frequency (the actual frequency emitted)
    \item $v$ = Speed of the wave in the medium (e.g., speed of sound, $\approx 343$ m/s in air)
    \item $v_o$ = Speed of the observer
    \item $v_s$ = Speed of the source
\end{itemize}

\subsection{Sign Convention: The Easy Way}
Don't memorize combinations! Just use one rule: \textbf{"Towards means higher frequency."}
\begin{enumerate}
    \item \textbf{Top signs (Observer):} The observer is in the numerator ($v_o$). To make $f_o$ higher (when the observer moves \textbf{towards} the source), we must add. So, use '$+$' for towards.
    \item \textbf{Bottom signs (Source):} The source is in the denominator ($v_s$). To make $f_o$ higher (when the source moves \textbf{towards} the observer), we must make the denominator smaller. So, use '$-$' for towards.
\end{enumerate}

\begin{center}
\begin{tabular}{|l|c|c|}
\hline
\textbf{Motion} & \textbf{Sign in Numerator ($v_o$)} & \textbf{Sign in Denominator ($v_s$)} \\
\hline
\textbf{Towards} each other & $+$ & $-$ \\
\textbf{Away} from each other & $-$ & $+$ \\
\hline
\end{tabular}
\end{center}

\subsection{Example 1: The Simple Case}
A car honking its horn at a frequency of \textbf{400 Hz} drives \textbf{towards} you at \textbf{20 m/s}. You are standing still. What frequency do you hear? (Speed of sound $v = 343$ m/s).

\begin{itemize}
    \item \textbf{Given:} $f_s = 400$ Hz, $v_s = 20$ m/s, $v_o = 0$ m/s (you are still), $v = 343$ m/s.
    \item \textbf{Motion:} Source moves \textbf{towards} a stationary observer. We want a higher frequency, so we use '$-$' in the denominator.
    \item \textbf{Calculation:}
    \[
    f_o = 400 \left( \frac{343 + 0}{343 - 20} \right) = 400 \left( \frac{343}{323} \right) \approx 424.8 \text{ Hz}
    \]
    As expected, the observed frequency is higher.
\end{itemize}

\hrulefill
\section{The Moving Reflector: A Two-Step Problem \quad \Large }

A moving reflector (like a wall, a truck, or a bat's prey) involves a \textbf{double Doppler shift}.
\begin{enumerate}
    \item \textbf{Step 1: The Reflector as a Moving Observer.} The reflector "hears" a shifted frequency from the original source.
    \item \textbf{Step 2: The Reflector as a Moving Source.} The reflector re-emits (reflects) this new frequency, and this wave travels back to the original source (which is now the final observer).
\end{enumerate}

\subsection{Example 2: Police Radar}
A police car at rest sends out a radar signal with a frequency $f_s$ of \textbf{10.5 GHz}. The signal bounces off a car moving \textbf{away} from the police car at \textbf{30 m/s}. What is the frequency of the reflected signal detected by the police car? (The speed of the radar wave is the speed of light, $c = 3 \times 10^8$ m/s).

\textbf{Step 1: Car as the Observer.} The car is moving \textbf{away} from the source (police).
\begin{itemize}
    \item $v_o = 30$ m/s, $v_s = 0$. We want a lower frequency, so use '$-$' for the observer.
    \item Frequency received by the car ($f_{\text{car}}$):
    \[
    f_{\text{car}} = f_s \left( \frac{c - v_o}{c + v_s} \right) = 10.5 \times 10^9 \left( \frac{3 \times 10^8 - 30}{3 \times 10^8 + 0} \right)
    \]
    \[
    f_{\text{car}} \approx 10.49999895 \text{ GHz}
    \]
\end{itemize}

\textbf{Step 2: Car as the Source.} The car now acts as a source, emitting $f_{\text{car}}$ while moving \textbf{away} from the observer (police).
\begin{itemize}
    \item $v_s = 30$ m/s, $v_o = 0$. We want a lower frequency, so use '$+$' for the source.
    \item Frequency received by police ($f_o$):
    \[
    f_o = f_{\text{car}} \left( \frac{c - v_o}{c + v_s} \right) = (10.49999895 \times 10^9) \left( \frac{3 \times 10^8 + 0}{3 \times 10^8 + 30} \right)
    \]
    \[
    f_o \approx 10.49999790 \text{ GHz}
    \]
\end{itemize}
The frequency has been shifted down twice.

\hrulefill
\section{Beats: The Sound of Interference \quad \Large}
When two sound waves of slightly different frequencies interfere, the loudness you hear varies periodically. This phenomenon is called \textbf{beats}. The number of loud-soft cycles per second is the \textbf{beat frequency}.

\subsection{Beat Frequency Equation}
The equation is very simple:
\[
f_{\text{beat}} = |f_1 - f_2|
\]
Where $f_1$ and $f_2$ are the frequencies of the two interfering waves. For example, if you play a 440 Hz note and a 442 Hz note at the same time, you will hear a "wa-wa" sound twice per second. The beat frequency is $442 - 440 = 2$ Hz.

\hrulefill
\section{Putting It All Together: The Complex Example \quad \Large}

Let's combine everything. A police car traveling at \textbf{25 m/s} is chasing a car traveling at \textbf{40 m/s} in the same direction. The police siren emits a sound at \textbf{500 Hz}.

\textbf{Question:} What is the beat frequency heard by the police officer between their own siren and the echo reflected from the chased car? (Use $v_{\text{sound}} = 343$ m/s).

This is a three-part problem:
\begin{enumerate}
    \item Find the frequency the chased car "hears" from the siren.
    \item Find the frequency the police car "hears" reflected back from the chased car.
    \item Calculate the beat frequency between the original siren sound (500 Hz) and the reflected echo.
\end{enumerate}

\textbf{Step 1: Frequency heard by the chased car (Observer).}
\begin{itemize}
    \item Police (Source) is moving \textbf{towards} the car at $v_s = 25$ m/s.
    \item Car (Observer) is moving \textbf{away} from the police at $v_o = 40$ m/s.
    \[
    f_{\text{car}} = f_s \left( \frac{v - v_o}{v - v_s} \right) = 500 \left( \frac{343 - 40}{343 - 25} \right) = 500 \left( \frac{303}{318} \right) \approx 476.4 \text{ Hz}
    \]
\end{itemize}

\textbf{Step 2: Frequency of the echo heard by the police.}
\begin{itemize}
    \item Now, the car is the \textbf{source}, moving \textbf{away} at $v_s = 40$ m/s and emitting at 476.4 Hz.
    \item The police car is the \textbf{observer}, moving \textbf{towards} the echo at $v_o = 25$ m/s.
    \[
    f_{\text{echo}} = f_{\text{car}} \left( \frac{v + v_o}{v + v_s} \right) = 476.4 \left( \frac{343 + 25}{343 + 40} \right) = 476.4 \left( \frac{368}{383} \right) \approx 458.2 \text{ Hz}
    \]
\end{itemize}

\textbf{Step 3: Calculate the beat frequency.}
\begin{itemize}
    \item The police officer hears two sounds: their own siren ($f_1 = 500$ Hz) and the reflected echo ($f_2 = 458.2$ Hz).
    \[
    f_{\text{beat}} = |f_1 - f_2| = |500 - 458.2| = \mathbf{41.8 \text{ Hz}}
    \]
\end{itemize}
The officer would hear a rapid flutter sound with a frequency of about 41.8 Hz.

\end{document}